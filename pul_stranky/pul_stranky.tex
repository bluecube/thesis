\documentclass[12pt,notitlepage,pdftex,a4paper]{article}
\usepackage{coolstyle}

\begin{document}

\section*{O čem to bude}

Budu se zabývat využitím GPS jako korekčního vstupu do Monte Carlo
lokalizace a její co nejdůkladnější využití.

Při jednoduchém použití GPS pro lokalizaci se využívá
odhad pozice, který poskytují GPS přijímače ve formátu WGS84.
U tohoto odhadu je ale problém s obtížně použitelnou chybou odhadu
(HDOP, VDOP a PDOP), malou frekvencí výstupů (cca 1 Hz) a také
není úplně zřejmé jestli jsou tyto výstupy z přijímače už nějak
filtrované a podobně.

Přístup, kterému bych se chtěl důkladněji věnovat je využití měření
pseudorange (změřené vzdálenosti od satelitů) úprava váhy vzorků v MCL
na základě těchto měření.
To sice zvýší výpočetní náročnost lokalizace, protože bude nutné pro
každý vzorek a každé měření pseudorange znovu upravovat váhu vzorku
pomocí většího množství floating-point operací, ale v kombinaci s
odometrií a případně i dalšími senzory to může přinést zlepšení
odhadu pozice oproti základnímu použití GPS.

Přístup k naměřeným hodnotám na nižší úrovni než WGS84 umožňuje
několik běžně dostupných GPS čipů, ve své práci se chci zaměřit
pouze na Sirf III.

Další možbostí rozšíření jsou differential GPS a carrier phase tracking.

\section*{Zdroje}

\begin{itemize}
\item \url{http://www.gdgps.net}

\item \url{http://home.tiscali.nl/~samsvl/}

\item Kinematic: \url{http://www.precision-gps.org/}

\end{itemize}

\end{document}
