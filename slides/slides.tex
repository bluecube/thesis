\documentclass[utf8]{beamer}

\usetheme{default}
\usecolortheme{seagull}
\usefonttheme{serif}
\setbeamertemplate{navigation symbols}{}

\usepackage{graphicx}
\usepackage{amsmath}
\usepackage{bm} % bold math symbols
\usepackage{tgpagella} % Gyre pagella -- based on palatino
\usepackage[english]{babel}
\usepackage[binary-units]{siunitx}

\input{math-defs.tex}

\title{Exploiting GPS in Monte Carlo Localization}
\author{Jakub Marek}


\newcommand{\imageframe}[2]{{
\setbeamertemplate{background}{
    \vbox to \paperheight{\vfil\hbox to \paperwidth{\hfil
    \includegraphics[width=\paperwidth,height=\paperheight,keepaspectratio]{#1} %%% nodep
    \hfil}\vfil}
    }
\setbeamercolor{background canvas}{bg=#2}
\begin{frame}[plain]
\end{frame}
}}


\begin{document}


\begin{frame}[plain]
    \titlepage
\end{frame}

\begin{frame}{Main Idea}
    \begin{itemize}
        \item Outdoor robots need position information
        \item Using low-level GPS measurements as an input to MCL
        \begin{itemize}
            \item Improve precision
            \begin{itemize}
                \item Individual measurements
                \item Cobination with other sensors
            \end{itemize}
            \item Improve error characterization
            \item Explore inner workings of GPS
        \end{itemize}
    \end{itemize}
\end{frame}

%%%% dependency img/01.jpg
%\imageframe{img/01.jpg}{black}

\begin{frame}{Why GPS}
    \begin{itemize}
        \item Worldwide absolute position
        \item No preparation required
        \item Cheap (?!)
    \end{itemize}

    \begin{itemize}
        \item Low precision from consumer GPS receivers
        \item Low measurement frequency
    \end{itemize}
\end{frame}

\begin{frame}{Why Monte Carlo Localization}
    \begin{itemize}
        \item Probabilistic localization algorithm
        \item Simple to understand and implement
        \begin{itemize}
            \item Simulate a swarm of possible robots
            \item Compare the actual and simulated sensor readings
        \end{itemize}
        \item Only requires probability of a measurement
        \item Arbitrary shapes of probability densities
    \end{itemize}

    \begin{itemize}
        \item More computationaly intensive than e.g. Kalman Filter
    \end{itemize}
\end{frame}

\begin{frame}{GPS -- Basic Principles}
    \begin{itemize}
        \item Measuring time of flight of radio signals
        \begin{itemize}
            \item Ideally: position = intersection of spherical surfaces
            \item In fact: 4D conical surfaces (3D positions + clock offsets)
        \end{itemize}
        \item Pseudorange
        \begin{itemize}
            %\item \(\rho = \speedoflight (\recrxtime - \svtxtime)\)
            \item Distance corresponding to time of flight
            \begin{itemize}
                \item Transmission time according to satellite clock
                \item Receipt time according to receiver clock
            \end{itemize}
        \end{itemize}
        \item Doppler measurements
        \begin{itemize}
            \item Relative velocities of satellite and receiver
        \end{itemize}
    \end{itemize}
\end{frame}

\begin{frame}{GPS with MCL -- Simple Approach}
    \begin{itemize}
        \item Most GPS receivers provide NMEA output
        \begin{itemize}
            \item Contains lat/lon/alt coordinates and HDOP
        \end{itemize}
        \item Can be used as a position input almost immediately
        \begin{itemize}
            %\item Rayleigh distribution, \( \sigma = \frac{\sqrt{(a \HDOP)^2 + b^2}}{\sqrt{2}} \)
            \item Horizontal position errors only
            \item No characterization of velocity error
        \end{itemize}
    \end{itemize}
\end{frame}

%%% dependency generated/wgs84-hdop-error.pdf
\imageframe{generated/wgs84-hdop-error.pdf}{white}

\begin{frame}{GPS with MCL -- Low Level Approach}
    \begin{itemize}
        \item Each pseudorange measurement as localization input
        \item Need to consider atmospheric delays
        \begin{itemize}
            \item Estimating low frequency errors as a part of the robot's state
        \end{itemize}
        \item Allows to use doppler measurements
    \end{itemize}
\end{frame}

\begin{frame}{GPS with MCL -- Low Level Approach}
    \begin{itemize}
        \item Simillar precision to the simple NMEA data
        \begin{itemize}
            \item Approximately \SI{7}{\meter} \(1\sigma\) errror for single measurement
            \item Room for improvement
        \end{itemize}
    \end{itemize}
\end{frame}

%%% dependency generated/pseudorange-errors.pdf
\imageframe{generated/pseudorange-errors.pdf}{white}

\begin{frame}{Comparison}

\end{frame}

\end{document}
