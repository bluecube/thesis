\documentclass[11pt,notitlepage,a4paper,pdftex]{memoir}

\usepackage[utf8]{inputenc} 
\usepackage[T1]{fontenc}

\usepackage{a4wide}
\usepackage{graphicx}
\usepackage{microtype}
\usepackage[
	left=4cm,
	right=3cm,
	top=3cm,
	]{geometry}
\usepackage[
	a4paper,
	unicode,
	colorlinks,
	linkcolor=black,
	urlcolor=black,
	citecolor=black,
	]{hyperref}
\usepackage{url}

%% Define a new style for url that will use a smaller font.
\makeatletter
\def\url@smallerstyle{%
  \@ifundefined{selectfont}{\def\UrlFont{\sf}}{\def\UrlFont{\footnotesize\ttfamily}}}
\makeatother
%% Now actually use the newly defined style.
\urlstyle{smaller}

% Text: garamond, palatino, bookman old style, baskerville
% Headings: futura Century gothic, gill sans, futura
% http://www.kronto.org/thesis/tips/packages.html

\begin{document}

\chapter{Introduction}

Localization is an important task for autonomous or semi-autonomous robots

Outdoor robots may navigate large areas without detailed maps available, this limits
the usefulness of some sensors used for indoor localization, like laser range finders.

Global positioning system, or GPS for short, is a satellite navigation system


Seeing its use in automotive industry, GPS seems to be an good candidate for
localization in robots operating in outdoor areas.
However if GPS should be used as one of the primary methods for robot localization,
consumer grade receivers don't offer enough precision.

Monte Carlo localization (MCL) is an algorithm that estimates position of a robot
based on several noisy measurements.
MCL is a type of bayesian filter that represents
%TODO: bayesian, nebo Bayesian?
the position estimation as a set of weighted samples.
One of the key advantages of MCL is the simplicity with which it \ldots
%TODO: PROČ to funguje je kouzlo, ale JAK to funguje se dá pochopit. Jak to napsat?

The goal of this work is therefore to gain as much precision as possible from low cost
GPS receivers in the framework provided by MCL algorithm.

The second chapter describes the GPS system and its operating principles.
The third chapter describes Monte Carlo Localization.
Fourth chapter talks about experiments with fusing of GPS data into
MCL localization.

\chapter{GPS}

\section{History of GPS}

\section{Principle of operation}


\chapter{Monte Carlo Localization}

\chapter{GPS in Monte Carlo Localization}

\chapter{Conclusion}

\end{document}
