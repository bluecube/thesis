\documentclass[11pt,notitlepage,a4paper,pdftex]{memoir}

\usepackage[utf8]{inputenc} 
\usepackage[T1]{fontenc}

\usepackage{a4wide}
\usepackage{graphicx}
\usepackage{microtype}
\usepackage[
	left=4cm,
	right=3cm,
	top=3cm,
	]{geometry}
\usepackage[
	a4paper,
	unicode,
	colorlinks,
	linkcolor=black,
	urlcolor=black,
	citecolor=black,
	]{hyperref}
\usepackage{url}

%% Define a new style for url that will use a smaller font.
\makeatletter
\def\url@smallerstyle{%
  \@ifundefined{selectfont}{\def\UrlFont{\sf}}{\def\UrlFont{\footnotesize\ttfamily}}}
\makeatother
%% Now actually use the newly defined style.
\urlstyle{smaller}

% Text: garamond, palatino, bookman old style, baskerville
% Headings: futura Century gothic, gill sans, futura
% http://www.kronto.org/thesis/tips/packages.html

\begin{document}

\chapter{Introduction}

Localization is an important task for autonomous or semi-autonomous robots

Outdoor robots may navigate large areas without detailed maps available, this limits
the usefulness of some sensors used for indoor localization, like laser range finders.
\marginpar{pure blah blah}

Seeing its use in automotive industry, GPS seems to be an good candidate for
localization in robots operating in outdoor areas.
However if GPS should be used as one of the primary methods for robot localization
several problems arise. 

The goal of this work is to find a way to deal with these problems.

The second chapter describes the GPS system and its operating principles.
The third chapter describes \marginpar{describes FUJ} Monte Carlo Localization.
Fourth chapter talks about \marginpar{FUJ} our \marginpar{our versus my ?} experiments
with fusing of GPS data into MCL localization \marginpar{FUJ}.

\chapter{GPS}

Global Postioning System is a worldwide

\section{History of GPS}

\section{Principle of operation}


\chapter{Monte Carlo Localization}

\chapter{GPS in Monte Carlo Localization}

\chapter{Conclusion}

\end{document}
