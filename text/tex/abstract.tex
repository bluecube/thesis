\noindent
Title: \thetitle\\
Author: \theauthor\\
Department / Institute: Department of Software Engineering\\
Supervisor of the master thesis: RNDr. David Obdržálek, Department of Software Engineering

\vspace{5mm}

\noindent
Abstract: This work presents two approaches for integrating data from a low
cost GPS receiver in a Monte Carlo localization algorithm.
First, an easily applicable method based on data in the standard NMEA protocol is shown.
Next an original algorithm utilizing lower level pseudorange measurements accessed in
binary receiver-specific format is presented.
A set of tools for analysis of GPS measurement errors on receivers
with \sirf chipset was implemented and a comparison of the approaches mentioned above is made.

\vspace{5mm}

\noindent
Keywords: robotics, localization, GPS, Monte Carlo localization, pseudorange, SiRF

\vspace{25mm}

%\begin{otherlanguage}{czech}
\noindent
Název práce: Použití GPS v Monte Carlo lokalizaci\\
Autor: \theauthor\\
Katedra / Ústav: Katedra softwarového inženýrství\\
Vedoucí diplomové práce: RNDr. David Obdržálek, Katedra softwarového inženýrství

\vspace{5mm}

\noindent
Abstrakt: Tato práce uvádí dva postupy pro integraci dat z~běžného GPS přijímače
v~Monte Carlo lokalizaci.
Jako první popisuje snadno použitelnou metodu založenou na použití dat přístupných ve standardním
protokolu NMEA.
Dále obsahuje originální algoritmus používající nízkoúrovňová měření pseudorange,
získávaná specializovaným protokolem daného přijímače.
Při tvorbě této práce vznikl balík nástrojů pro analýzu chyb měření GPS přijímačů
s chipsetem \sirf a zmíněné metody byly s jeho pomocí porovnány.

\vspace{5mm}

\noindent
Klíčová slova: robotika, lokalizace, GPS, Monte Carlo lokalizace, pseudorange, SiRF
%\end{otherlanguage}

\cleartorecto

