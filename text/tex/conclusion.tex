\chapter{Conclusion}
\label{chap:conclusion}

The goal of this thesis was to explore methods of utilizing of a low cost GPS receiver
for outdoor robot localization in the framework of the Monte Carlo localization algorithm.
Two approaches to perform this task were selected.

Firstly, the \enquote{obvious} method of interfacing to the GPS receiver with the
standard NMEA protocol was evaluated.
According to this approach the complete fixes in WGS84 coordinates are used,
with latitude, longitude and HDOP being the only values used in the localization.
Error model was constructed from a set of experimentally measured data,
giving one sigma precision of approximately \SI{8}{\meter}.

Secondly, a new method was developed, working with GPS one level of abstraction deeper.
This method makes use of the individual pseudorange and velocity measurements
and compares the reported and expected distances and relative velocities between
the receiver and the GPS satellites.
To obtain the raw measurements for this method, communication with the GPS
receiver had to be performed using the proprietary binary protocol,
since such detailed data is not available in the standard NMEA messages.
Again, an error model was calculated from recorded data.
In the current state, this method offers a precision comparable to the simple
approach, showing room for improvement in follow-up work.
This provides position estimates free of the assumptions made in the NMEA data
and offers superior error characterization.

To construct the error models and to visualize the data, a set of tools was implemented.
These consist of a library for communication with \sirf GPS receivers
and a group of individual scripts for experimenting with various aspects of the GPS signals.
During development of these tools we successfully overcame a number of problems caused mainly
by the lack of documentation of the \sirf chip operation details.

In the current state, this work could be of benefit to the designers of
outdoor robots, allowing them to simply use the first method for modelling the
error of GPS in NMEA mode.
Our second method provides the robot designers with comparable precision,
velocity processing, more detailed characteristics of the position error
and a framework allowing insight into the localization process and further
precision improvements.
On top of that we provide a framework for experiments with SiRF receivers
for anyone interested.
We also hope this text might serve as another piece of the fragmented
information about the \sirf chipset on the internet.
