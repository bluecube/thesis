\chapter{Conclusion}
\label{chap:conclusion}

A goal of this work was to explore methods of utilizing a low cost GPS receiver
for outdoor robot localization in the framework of the Monte Carlo localization algorithm.
Two approaches to perform this task were selected.

First, the \enquote{obvious} method of interfacing to the GPS receiver with the
standard NMEA protocol was evaluated.
According this approach the complete fixes in WGS84 coordinates are utilized,
with latitude, longitude and HDOP as the only values used in the localization.
Error model was constructed from a set of experimentally measured data,
giving one sigma precision of approximately \SI{10}{\meter}.

A second method, working with GPS one level of abstraction lower, was proposed and developed,
making use of the individual pseudorange and velocity measurements
of the GPS and comparing the reported and expected distances and relative velocities
between the receiver and the GPS satellites.
Again an error model was calculated from recorded data and,
at the current state, the second method offers a precision comparable to the simple
approach, showing room for improvement in follow up work.

To construct the error models and to visualize the data, a set of tools was developed.
These consist of a library for communication with \sirf GPS receivers,
and a group of individual scripts for experimenting with various aspects of the GPS signals.
During development of these tools we overcame a number of obstacles caused mainly
by lack of documentation of the detailed workings of the \sirf chip.

There is a large area for improvements and further work, mainly, as mentioned
above, in the area of precision improvements of the low level algorithm,
but admittedly also in testing and verification of the method's robustness.

Even in its current state, this work could be of benefit to the designers of
outdoor robots, allowing the either to simply use the first model, or, if they
wish to dig deeper into the GPS system, to improve the localization precision
and error characterization with the second method.
The tools are made available on the internet to allow anyone interested in this
topic to experiment on their own.
Other than that, this thesis might serve as another piece of the fragmented
informations about the \sirf chipset on the internet.
