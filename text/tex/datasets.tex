\chapter{Datasets}
\label{chap:datasets}

Three sets of measured GPS data were recorded for this work.
All of them were recorded using GlobalSat BR-355 serial GPS receiver with \sirf chip
in the SiRF binary protocol.

\begin{itemize}

\item
First dataset,  contains approximately one month of recorded SiRF data.
During this period the receiver was stationary inside a room and had a good signal 
reception.
It was used for modeling of the GPS measurements errors in \Cref{chap:gps-and-mcl}.

The dataset contains \num{\monthDatasetCycles} measurement cycles and
\num{\monthDatasetMeasurements} pseudorange measurements.

For quick testing, we also keep a subset of this recording containing the
first night of the dataset and of first week.

On the DVD the main recording of dataset is located in \verb=/data/month.recording2=,
the shortened versions can be found in \verb=/data/one_night.recording2= and in
\verb=/data/one_week.recording2=.

\item
The second dataset consists of two recordings made during
the Robotour competition \cite{robotour} on a Roboauto Karlík \cite{karlik} robot\todo{Better citations?}.
%These data were used as an input for Monte Carlo Localization for performance comparison.

In total the dataset contains \num{\karlikDatasetCycles} GPS measurement cycles with
\num{\karlikDatasetMeasurements} pseudorange measurements, accompanied by sensor recordings from Karlík.

Recordings for this dataset can be found in files \verb=/data/roboauto1.recording2=
and \verb=/data/roboauto2.recording2=.
Karlík's log files are located in the directory \verb=/data/roboauto/=.

\item
The last dataset contains approximately
30 minutes recorded with stationary receiver with poor signal quality.

The recording is in the file \verb=/data/30_minutes_weak_signal.recording2=.

\end{itemize}

\Cref{fig:hdop-hist} contains histograms showing the distribution of HDOP values in the datasets.
%Missing from the plots are outliers with HDOP more than 10.

\begin{figure}[htp]
	\centering
	\noindent\makebox[\textwidth]{
	\includegraphics{generated/hdop-hist.pdf}
	}
	\caption{Histograms of HDOP values for our data sets.}
	\label{fig:hdop-hist}
\end{figure}


