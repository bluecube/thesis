\chapter{GPS and Monte Carlo Localization}
\label{chap:gps-and-mcl}

This chapter discusses using GPS data as an input to to Monte Carlo localization.

As discussed in \cref{chap:mcl}, Monte Carlo localization accepts two kinds of
inputs -- prediction and correction.
Of these two we will consider only correction inputs as GPS doesn't provide any
a priori data about motion of the robot.

First we will concentrate on the approach based on high level data that are available
from the standard WGS84 protocol,
second we will use individual pseudorange measurements each as an independent
correction input to MCL.

\section{Position Domain Integration}
Integrating the measurements in position domain means processing the complete
fixes in ECEF reference frames (or WGS84 converted to ECEF, see \cref{sec:ref-frames})
and treating the GPS receiver as a source of absolute position and velocity information.

This approach has the advantage of being very simple, because most of the work
is done in the dedicated hardware of the GPS receiver.
On the other hand information are discarded when the fix is converted to the simple
latitude / longitude / HDOP format, which may decrease the accuracy of the
localization.
Also consumer-grade GPS receivers often employ filters tuned to specific properties
of the platform intended to carry the unit for example ignoring speeds
below a certain threshold or limiting allowed accelerations.
\velocitytodo

\subsection{Sample Format}
For position domain integration we work in a two dimensional coordinate
system on a surface of Earth, with X axis going East and Y axis going North.
WGS84 inputs are transformed to this reference frame using orthogonal projection
this reported position and HDOP are used to modify weights of the samples.
\velocitytodo

No other variables are required in the robot's state for this method.
This makes it possible to seamlessly integrate it into existing localization framework.
Adding another more state, however might improve the accuracy, one such option
is discussed in \cref{sec:position-domain-correlation}.

\subsection{Sensor Model}
\label{sec:wgs84-hdop-error}

\begin{figure}[htp]
	\centering
	\noindent\makebox[\textwidth]{
	\includegraphics{plots/wgs84-hdop-error.pdf}
	}
	\caption{Measurement errors vs. HDOP.}
	\label{fig:wgs84-hdop-error}
\end{figure}

Position domain integration has no requirements on the prediction phase
and only 

For modelling the error we basically follow \cite{www-wilson}.

In this case we are operating only in two dimensions.
The horizontal component of the position error  \(\vect{\delta R}\) is modelled using Rayleigh distribution
parameterized with HDOP of the measurement:
\begin{equation}
	\Prob(\norm{\vect{\delta R}} < x \mid \HDOP) =
		1 - e^{-x^2/2\sigma(\HDOP)^2}
\end{equation}

Parameter \(\sigma(\HDOP)\) of the distribution is chosen to fit the DRMS values using a maximum likehood estimate
\begin{equation}
	\sigma(\HDOP) = \frac{\mathrm{DRMS}(\HDOP)}{\sqrt{2}}
\end{equation}

Experimentally measured data were fitted to the theoretical linear model and to
the non-linear model from \cite{www-wilson}.
Both were fitted to the data using least squares weighted with counts of samples
for each HDOP, resulting in the expressions
\begin{equation}
\mathrm{DRMS}(\HDOP) = \num{6.255} \HDOP
\end{equation}
and
\begin{equation}
\mathrm{DRMS}(\HDOP) = \sqrt{(\num{4.941}\HDOP)^2 + \num{3.568}^2}
\end{equation}

\Cref{fig:wgs84-hdop-error} is a visualisation of fitting of these two models,
blue dots representing the measured data, yellow dots the DRMS values.
Green line shows the theoretical linear model and red curve is the non linear model.
Limited resolution of the HDOP values in input data is visible in the plot.
Another interesting fact is, that only relatively low HDOP values were encountered.
\Cref{fig:hdop-hist} shows this in a more pronounced way.

To obtain the final sensor model, we must express the PDF of Rayleight distribution:
\begin{equation}
    f(x) = \frac{x}{\sigma^2}e^{-x^2/2\sigma^2}
\end{equation}

\Cref{algo:gps-position-domain} shows a pseudo code implementation of a function
that provides a sensor model to MCL algorithm from \cref{chap:mcl} based on the
fitted non-linear model.

\begin{figure}[tp]
\begin{algorithm}[H]
    \SetKwFunction{project}{project}
    \SetKwFunction{abs}{abs}
    \SetKwFunction{exp}{exp}

    \SetKwData{sample}{sample}
    \SetKwData{projected}{projected}
    \SetKwData{observation}{observation}
    \SetKwData{hdop}{HDOP}
    \SetKwData{latLon}{latLon}
    \SetKwData{pos}{position}
    \SetKwData{positionError}{positionError}
    \SetKwData{drms}{DRMS}
    \SetKwData{drmsSquared}{\(\drms^2\)}

    \function{observationProbabilityGPS}{
        \KwIn{sample, observation}
        \KwOut{probability}
        \BlankLine
        \projected \assign \project{\observation.\latLon}\;
        \hdop \assign \observation.\hdop\;
        \BlankLine
        \positionError \assign \abs{\projected \(-\) \sample}\;
        \drmsSquared \assign \( ( \num{4.941}\hdop )^2 + \num{3.568}^2 \)\;% (\num{4.941}\HDOP)^2 + \num{3.568}^2 )\;
        \BlankLine
        \Return 2 * \positionError / \drmsSquared * \exp{-\(\positionError^2\)/\drmsSquared} \;
    }
\end{algorithm}
\caption{Correction algorithm for position domain integration of GPS measurements}
\label{algo:gps-position-domain}
\end{figure}

\subsection{Correlation of Position Errors}
\label{sec:position-domain-correlation}
A problem that is hard to avoid with this approach is that position errors of the
receiver output are correlated.
This is in part because errors of the individual satellite measurements are correlated
(which is discussed in \cref{sec:measurement-domain-correlation}), but another major reason
is the \enquote{inertia} added by the Kalman filter.
This kind of correlation obviously breaks the independence assumptions of Markov
localization defined in \cref{sec:markov-assumptions}.

An attempt at mitigating this could be done by estimating the error as part of
the robot's state, possibly removing atmospheric effects and some of the low pass
filtering properties of the Kalman filter in the receiver.
These options, however, will not be explored in this work.
\todo{possibilities}\todo{Mention if we end up doing this in measurement domain}

\section{Measurement Domain Integration}
\label{sec:measurement-domain}

\begin{itemize}
\item Using each pseudorange measurement
\item Maybe better localization, no assumptions about vehicle properties
\end{itemize}

\subsection{Static Environment Assumption}
\label{sec:gps-mcl-static-env}
Monte Carlo Localization assumes that the environment is static, but we need to
localize according to moving satellites.
To avoid this problem we will hide satellite motion into the robot state and sensor model.

\subsection{Robot State}


\subsection{Error distribution}
Normal distribution (???)

\subsubsection{Error correlation}
\label{sec:measurement-domain-correlation}
Keep atmospheric delays as part of the localization state?

\subsection{Preprocessing}
\begin{itemize}
\item ???
\end{itemize}

\subsection{Initial estimate}
\begin{itemize}
\item starting as position domain, later switching to measurement domain
\end{itemize}

\subsection{Simplifications}
\begin{itemize}
\item Ignoring altitude (???)
\item Additional 1D kalman for altitude (???)
\item Additional 1D kalman for receiver clock error (???)
\end{itemize}
