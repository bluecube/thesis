\chapter{GPS and Monte Carlo Localization}
\label{chap:gps-and-mcl}

This chapter discusses using GPS data as an input to to Monte Carlo localization.

As discussed in \cref{chap:mcl}, Monte Carlo localization accepts two kinds of
inputs -- prediction and correction.
Of these two we will consider only correction inputs as GPS doesn't provide any
a priori data about motion of the robot.

First we will concentrate on the approach based on high level data that are available
from the standard WGS84 protocol,
second we will use individual pseudorange measurements each as an independent
correction input to MCL.

\section{Position Domain Integration}
Integrating the measurements in position domain means processing the complete
fixes in ECEF reference frames (or WGS84 converted to ECEF, see \cref{sec:ref-frames})
and treating the GPS receiver as a source of absolute position and velocity information.

This approach has the advantage of being very simple, because most of the work
is done in the dedicated hardware of the GPS receiver.

\subsection{Sensor Model}
\label{sec:wgs84-hdop-error}

\begin{figure}[htp]
	\centering
	\noindent\makebox[\textwidth]{
	\includegraphics{plots/wgs84-hdop-error.pdf}
	}
	\caption{Measurement errors vs. HDOP.}
	\label{fig:wgs84-hdop-error}
\end{figure}

For modelling the error we basically follow \cite{www-wilson}.

WGS84 inputs are transformed to robot's local coordinate system (see
\cref{sec:impl-coordinates}) and the reported position and HDOP are used
to modify weights of the samples\todo{What about velocities?}.

In this case we are operating only in two dimensions.
The horizontal component of the position error  \(\vect{\delta R}\) is modelled using Rayleigh distribution
parameterized with HDOP of the measurement:
\begin{equation}
	\Prob(\norm{\vect{\delta R}} < x \mid \HDOP) =
		1 - e^{-x^2/2\sigma(\HDOP)^2}
\end{equation}

Parameter \(\sigma(\HDOP)\) of the distribution is chosen to fit the DRMS values using a maximum likehood estimate
\begin{equation}
	\sigma = \frac{\mathrm{DRMS}}{\sqrt{2}}
\end{equation}

\Cref{fig:wgs84-hdop-error} plots position error versus HDOP value.
Blue dots represent the measured data, yellow dots the DRMS values.
Green line is the fitted theoretical linear model
\begin{equation}
\mathrm{DRMS}(\HDOP) = \num{6.255} \HDOP
\end{equation}
and red curve is the fitted non linear model from \cite{www-wilson}
\begin{equation}
\mathrm{DRMS}(\HDOP) = \sqrt{(\num{4.941}\HDOP)^2 + \num{3.568}^2}
\end{equation}
Both of these models were fitted to the data using least squares weighted with counts of samples
for each HDOP.

Limited resolution of the HDOP values in input data is visible in the plot,
another interesting fact is, that only relatively low HDOP values were encountered.
This can be seen in \cref{fig:hdop-hist}.

A problem that is hard to avoid with this approach is that position errors of the
receiver output are correlated.
This is in part because errors of the individual satellite measurements are correlated
(which is discussed in \ref{sec:measurement-domain-correlation}), but another major reason
is the \enquote{inertia} added by the Kalman filter.
This kind of correlation obviously breaks the independence assumptions of Markov
localization defined in \ref{sec:markov-assumptions}, but in this case we chose to ignore
the problem. \todo{Or not.}

\section{Measurement Domain Integration}
\label{sec:measurement-domain}

\begin{itemize}
\item Using each pseudorange measurement
\item Maybe better localization, no assumptions about vehicle properties
\end{itemize}

\subsection{Static Environment Assumption}
\label{sec:gps-mcl-static-env}
Monte Carlo Localization assumes that the environment is static, but we need to
localize according to moving satellites.
To avoid this problem we will hide satellite motion into the robot state and sensor model.

\subsection{Error distribution}
Normal distribution (???)

\subsubsection{Error correlation}
\label{sec:measurement-domain-correlation}
Keep atmospheric delays as part of the localization state?

\subsection{Preprocessing}
\begin{itemize}
\item ???
\end{itemize}

\subsection{Initial estimate}
\begin{itemize}
\item starting as position domain, later switching to measurement domain
\end{itemize}

\subsection{Simplifications}
\begin{itemize}
\item Ignoring altitude (???)
\item Additional 1D kalman for altitude (???)
\item Additional 1D kalman for receiver clock error (???)
\end{itemize}
