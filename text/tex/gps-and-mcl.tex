\chapter{GPS Used in Monte Carlo Localization}
\label{chap:gps-and-mcl}

For an autonomous outdoor robot a GPS seems to be a good candidate for a source of
absolute position information.
Positions reported by low cost GPS receivers, however, suffer from large errors,
which make it impractical to be used as a main localization sensor.
Monte Carlo localization is an algorithm that has the ability to deal with noisy
inputs and is relatively simple to both understand and implement.
Moreover only a relatively simple model of the input sensor is necessary to work
with MCL, enabling it to calculate the GPS position solutions without requiring
a lot of theory otherwise necessary to use GPS at the lower level.

In this chapter we will show two approaches to using GPS receivers with MCL.
First we will concentrate on the approach based on high level data that are available
from the standard WGS84 protocol.
Next we will show an original algorithm that utilizes
the individual pseudorange measurements each as an independent correction input.

This chapter concentrates on tasks common to all GPS receivers,
problems and implementation decisions encountered with the SiRF receiver
during the implementation part are addressed in \cref{chap:implementation}.

GPS augmentation systems (like EGNOS or WAAS) were not used for tests in this
to keep the sensor models simple.
Extending the algorithms provided here to utilize GPS augmentation is an opportunity
for further work, although it should not introduce dramatic changes.

\section{Position Domain Integration}
\label{sec:position-domain}
Integrating the measurements in position domain means processing the complete
fixes and treating the GPS receiver as a source of absolute position and velocity information.
This processing can be performed in ECEF reference frames, or, as is the case in this work,
in 2D plane.

This approach has the advantage of being very simple, because most of the work
is done in the dedicated hardware of the GPS receiver.
On the other hand information are discarded when the fix is converted to the simple
latitude / longitude / HDOP format, which may decrease the accuracy of the
localization.
Also consumer-grade GPS receivers often employ filters tuned to specific properties
of the platform intended to carry the unit.
These may for example ignore speeds below a certain threshold or limit allowed accelerations.

It would also be possible for position domain integration to utilize the speed information
from the GPS (NMEA message GPRMC contains velocity in knots and track angle \cite{depriest}).
For this work, however, we choose not to employ it, to avoid the complexity of converting
track course and speed to reference frames usable for robot navigation.

\subsection{Robot State}
For position domain integration we work in a two dimensional coordinate
system on a surface of Earth, with X axis going East and Y axis going North.
WGS84 inputs are transformed to this reference frame using orthogonal projection
this reported position and HDOP are used to modify weights of the samples.
Details of this can be found in \cref{sec:impl-coordinates}.

No other variables are required in the robot's state for this method.
This makes it possible to seamlessly integrate it into existing localization framework.
Adding  more state, however might improve the accuracy, one such option
is discussed in \cref{sec:position-domain-correlation}.

\subsection{Sensor Model}
\label{sec:wgs84-hdop-error}

\begin{figure}[htp]
	\centering
	\noindent\makebox[\textwidth]{
	\includegraphics{generated/wgs84-hdop-error.pdf}
	}
	\caption{Measurement errors vs. HDOP.}
	\label{fig:wgs84-hdop-error}
\end{figure}

Since the position domain integration as used here has no persistent variables in the
robot state, it does not need any operations during the prediction step of MCL.

Our sensor model for the correction phase closely follows \cite{www-wilson}.
We are operating only in two dimensions and
the horizontal component of the position error  \(\vect{\delta R}\) is modelled using Rayleigh distribution
parameterized with HDOP of the measurement:
\begin{equation}
	\Prob(\norm{\vect{\delta R}} < x \mid \HDOP) =
		1 - e^{-x^2/2\sigma(\HDOP)^2}
\end{equation}

Parameter \(\sigma(\HDOP)\) of the distribution is chosen to fit the DRMS values using a maximum likehood estimate
\begin{equation}
	\sigma(\HDOP) = \frac{\mathrm{DRMS}(\HDOP)}{\sqrt{2}}
\end{equation}

The experimentally measured data were projected to a plane and distance errors
were calculated as a difference of the point position from a mean position.
These errors were then fitted to the theoretical linear model and to
the non-linear model from \cite{www-wilson}.

Both were fitted to the data using least squares weighted with counts of samples
for each HDOP, resulting in the expressions
\begin{equation}
\mathrm{DRMS}(\HDOP) / \si{\meter} = \num{\sigmaUere} \HDOP
\end{equation}
and
\begin{equation}
\mathrm{DRMS}(\HDOP) / \si{\meter} = \sqrt{(\num{\WilsonParamA}\HDOP)^2 + \num{\WilsonParamB}^2}
\end{equation}

\Cref{fig:wgs84-hdop-error} is a visualisation of fitting of these two models,
blue dots representing the measured data, yellow dots the DRMS values.
The green line shows the theoretical linear model and the red curve is the non linear model.
Limited resolution of the HDOP values in input data is visible in the plot.
This figure also shows the precision available when using simple NMEA data --
DRMS error of roughly \SI{8}{\meter}.
Another interesting fact is to note, that only relatively low HDOP values were encountered.
\Cref{fig:hdop-hist} shows this in a more pronounced way.

To obtain the final sensor model, we must express the PDF of Rayleigh distribution:
\begin{equation}
    f(x) = \frac{x}{\sigma^2}e^{-x^2/2\sigma^2}
\end{equation}

\Cref{algo:gps-position-domain} shows a pseudo code implementation of a function
that provides a sensor model to MCL algorithm from \cref{chap:mcl} based on the
fitted non-linear model.

\begin{figure}[tp]
\begin{algorithm}[H]
    \SetKwFunction{project}{project}
    \SetKwFunction{abs}{abs}
    \SetKwFunction{exp}{exp}

    \SetKwData{sample}{sample}
    \SetKwData{projected}{projected}
    \SetKwData{observation}{observation}
    \SetKwData{hdop}{HDOP}
    \SetKwData{latLon}{latLon}
    \SetKwData{pos}{position}
    \SetKwData{positionError}{positionError}
    \SetKwData{drms}{DRMS}
    \SetKwData{drmsSquared}{\(\drms^2\)}

    \function{observationProbabilityGPS}{
        \KwIn{sample, observation}
        \KwOut{probability}
        \BlankLine
        \projected \assign \project{\observation.\latLon}\;
        \hdop \assign \observation.\hdop\;
        \BlankLine
        \positionError \assign \abs{\projected \(-\) \sample}\;
        \drmsSquared \assign \( ( \num{\WilsonParamA}\hdop )^2 + \num{\WilsonParamB}^2 \)\;
        \BlankLine
        \Return 2 * \positionError / \drmsSquared * \exp{-\(\positionError^2\)/\drmsSquared} \;
    }
\end{algorithm}
\caption{Correction algorithm for position domain integration of GPS measurements}
\label{algo:gps-position-domain}
\end{figure}

\subsection{Correlation of Position Errors}
\label{sec:position-domain-correlation}
A problem that is hard to avoid with this approach is that position errors of the
receiver output are correlated.
This is in part because errors of the individual satellite measurements are correlated,
but another major reason
is the \enquote{inertia} added by the Kalman filter.
This kind of correlation obviously breaks the independence assumptions of Markov
localization defined in \cref{sec:markov-assumptions}.

An attempt at mitigating this could be done by estimating the error as part of
the robot's state, possibly removing atmospheric effects and some of the low pass
filtering properties of the Kalman filter in the receiver.
These options, however, will not be explored in this work.

\section{Measurement Domain Integration}
\label{sec:measurement-domain}

Measurement domain integration takes each pseudorange measurement as an input
and combines it with other sensor data.

In Monte Carlo localization, the GPS data are used in a similar way as
for example ultrasound ranging beacons would be used,
taking each measured pseudorange after a correction for measurement errors
and after clock correction modifying the sample weight based on the measured distance
compared to expected distance.
GPS, of course, has the additional complexity of the receiver clock offset, which
has to be known in order to transform pseudoranges to measured geometric ranges.

The main reason for attempting to use the GPS in this way is to add more information
into the localization process and to have more control over assumptions made during
processing of the GPS data.

To improve accuracy, we also track residual errors for each individual satellite.
Main causes for these errors are in ionosphere, but this treatment should be also
capable of alleviating other errors that change slowly over time, like satellite
clock offsets and partially also ephemeris errors.

To improve the precision of the localization, we also estimate residual errors of each satellite,
covering all of the slowly changing errors that are present
in each satellite's transmission and are not explicitly corrected elsewhere in the process.
This includes satellite clock inaccuracies, unmodeled remains of ionospheric and
tropospheric errors and to some extent also satellite ephemeris errors.
While this is not strictly necessary and the localization can work even when only
estimating receiver clock offset and drift, this significantly improves precision.
In the experimental data estimation of residual errors decreased one sigma pseudorange error
from \SI{\pseudorangeErrorNaiveSigma}{\meter} to \SI{\pseudorangeErrorSigma}{\meter}.

As with the Position domain approach, the model was derived from
measurements in a month long data set (see \cref{chap:datasets}).
During the test period the receiver was stationary and
the \enquote{true} position for checking the errors was established as an average of ECEF
positions reported by the receiver's internal software.
Clock offsets and drifts, which do not stay constant during the experiment,
were approximated by fitting a linear function to a sliding window of pseudorange
measurements (details are discussed in \cref{sec:impl-clock-offsets}).

\subsection{Robot State}
Unlike position domain integration, this method requires specific
way of storing the position and velocity of the robot and
additional state data in the samples used in MCL.

First of all, all of the GPS calculations are performed in ECEF coordinate system,
which means that it is convenient to store the position and velocity of the robot in this coordinate system.
Next, we need to estimate a receiver clock offset together
with position, so this is another variable which must exist in the state.
To work with velocities, the state variables will also have to contain robot velocities in
ECEF reference frame and the receiver clock drift.

The residual errors are modelled as residual clock offset and residual clock drift
one pair of variables for each satellite.

\subsection{Preprocessing}
Relatively large amount of work is dedicated to receiver-specific preprocessing
of the GPS data.
In short, this consists of merging ephemeris to the measurements, removing
any inconsistencies in data caused by the
receiver and converting the data to sequences of pseudoranges and Doppler measurements.
These steps are discussed in \cref{sec:impl-sirf-measurements,sec:impl-sirf-jumps,sec:impl-sirf-ephemeris}.

\subsection{Motion Model}
Since GPS utilizes additional state variables, we needs to specify the motion
model that governs these variables during the prediction phase of MCL.

Both the receiver clock drift and the residual clock drifts are modelled as a Gaussian random walk,
with the translation distance distribution
\begin{equation}
\normalDistribution(\mu = \SI{\clockDriftRateMu}{\meter}, \sigma = \SI{\clockDriftRateSigma}{\meter})
\end{equation}
for the receiver clock drift and 
\begin{equation}
\normalDistribution(\mu = \SI{\residualClockDriftRateMu}{\meter}, \sigma = \SI{\residualClockDriftRateSigma}{\meter})
\end{equation}
for the residual clock drifts.
Clock offsets are obtained by integrating the clock drifts.
\todo{Resetting residual errors for long unseen satellites}
\todo{These distributions are not gaussian in fact.}

\Cref{fig:clock-drift-derivation,fig:residual-clock-drift-derivation}
contain histograms of clock drifts derivation and residual
clock drifts derivation.

\begin{figure}[htp]
	\centering
	\missingfigure{Derivation of clock drifts.}
	\caption{Derivation of clock drifts.}
	\label{fig:clock-drift-derivation}
\end{figure}

\begin{figure}[htp]
	\centering
	\missingfigure{Derivation of residual clock drifts.}
	\caption{Derivation of residual clock drifts.}
	\label{fig:residual-clock-drift-derivation}
\end{figure}


\subsection{Sensor Model}
Sensor model of the individual measurements calculates the probability of this
measurement being observed, conditioned by the robot state.

\subsubsection{Ephemeris}
We treat position and velocity of the satellite transmitting
the current ranging message as part of the measurement.
In reality these information are separate in the transmission, but
this approach works without problems when processing the GPS data from the receiver online,
because the GPS contains current best estimate of the ephemeris periodically repeated
in the broadcast navigation data.

If only the pseudorange measurements were available, we would need to access the
ephemeris data from the time of transmission.
This might in fact improve quality of the localization if precise ephemeris
was used instead (see \cref{sec:gps-ephemeris}).

\subsubsection{Corrections}
Pseudoranges and velocities reported by the receiver must be compensated before
use for receiver and satellite clock offset and drift and possibly also for other
sources of errors.

For pseudoranges, the clock offsets are compensated according to \cref{eq:pseudorange4}.
Other errors should be compensated as described in \cref{sec:gps-error-sources}
(our implementation uses only tropospheric corrections, see \cref{sec:impl-corrections}.

In a similar way, velocities must be compensated for clock drifts 
as in \eqref{eq:doppler2}.

\subsubsection{Expected Distance and Velocity}
To calculate the probability, the algorithm needs to know the distance and relative
velocity between the satellite and the potential position of the robot represented
by the sample.
Expected distance is simply a length of the vector between the position of the SV
and the sample.
Expected velocity is calculated according to equation \eqref{eq:doppler-velocity}.
Position and velocity errors are then calculated as a difference between the expected
and real corrected pseudorange or velocity.

\subsubsection{Thresholds}
\label{sec:gps-thresholds}
To avoid outliers that frequently appear in the received data (from several hundreds
of meters to thousands of kilometers far from the expected measurement), we
ignore measurements appearing further than a given threshold from the expected position.
The decision whether to throw away a given measurement is made after all available corrections
are applied.
Values used for the threshold are \SI{\pseudorangeThreshold}{\meter} for pseudorange
and \SI{\velocityThreshold}{\meter\per\second} for the velocity measurements.

To handle the case of catastrophically wrong estimate in the Monte Carlo localization,
thresholded values are assigned probability of \num{\pseudorangeThresholdProbability} for
pseudoranges and \num{\velocityThresholdProbability} for velocities.
These values correspond to the probabilities of a measurement being outside the
threshold in the experimental data.

\subsubsection{Error Distribution}
To obtain probability of a measurement that is within the threshold, we evaluate
PDF of normal distribution.
For pseudoranges it is
\begin{equation}
\normalDistribution(\mu = \SI{\pseudorangeErrorMu}{\meter\per\second\squared}, \sigma = \SI{\pseudorangeErrorSigma}{\meter\per\second\squared})
\end{equation}
and for the velocities
\begin{equation}
\normalDistribution(\mu = \SI{\velocityErrorMu}{\meter\per\second\squared}, \sigma = \SI{\velocityErrorSigma}{\meter\per\second\squared})
\end{equation}

\Cref{fig:pseudorange-hist,fig:velocity-hist} show the histograms of pseudorange
and velocity measurements.
It is possible, that the odd shape of velocity measurements probability distribution
is caused by an error in the SiRF binary protocol manual, however we didn't manage
to confirm this.
This effect is discussed in slightly more detail in \cref{sec:impl-carrier-freq},
but meanwhile we are using the mentioned normal distribution as a very simple
(and slightly incorrect) workaround.

\Cref{fig:pseudorange-errors} contains a plot of pseudorange errors in time.
In this plot blocks with larger errors are apparent.
Detecting these blocks and switching to different error distribution inside them
is one of the major improvements to be made in follow up work.

\begin{figure}[htp]
	\centering
	\missingfigure{Derivation of residual clock drifts.}
	\caption{Histogram of pseudorange errors.}
	\label{fig:pseudorange-hist}
\end{figure}

\begin{figure}[htp]
	\centering
	\missingfigure{Derivation of residual clock drifts.}
	\caption{Histogram of velocity errors.}
	\label{fig:velocity-hist}
\end{figure}

\begin{figure}[htp]
	\centering
	\missingfigure{Pseudorange measurement errors of a single satellite.}
	\caption{Pseudorange measurement errors of a single satellite.}
	\label{fig:pseudorange-errors}
\end{figure}


\subsection{Initialization}
The intended use of this algorithm is to improve position estimate when using
a consumer grade GPS receiver, so we are not limited by the need to bootstrap the estimates.
For localizing the robot globally, we can initialize the position
estimate from the solution provided by the receiver's software.
The initial position estimate can be modelled as discussed in \cref{sec:position-domain}.

Clock offsets and drifts can be initialized from the data provided by the
receiver, although there is no standardized way of doing this.
As an alternative, clock offsets can be estimated from the positions that were obtained in
the previous step.
To do this we can assume that the next pseudorange measurement is error-free
and obtain clock correction from equation \eqref{eq:pseudorange4}.
Equation \eqref{eq:doppler2} can be used in a similar fashion to get receiver clock drift.

Since the residual clock drifts and offsets are not necessary for the algorithm,
they can be initialized to zero.

\subsection{Algorithm}
\Cref{algo:gps-measurement-domain} sums up the measurement domain integration
algorithm for Monte Carlo localization.

The motion model is implemented in function \predictGPS{}.
It is expected to be called from the function \sampleFromActionModel{} in \cref{algo:mcl},
as a part of the prediction step.
This function advances the clock drift by a random amount
and integrates the current drift to form a new clock offset,
both for the receiver clock drift and offset and for the
clock drifts and offsets of individual satellites.
Function \timeFunc{} returns the value of the system performing the localization
and is not required to be synchronized to GPS time in any way.

The sensor model in function \observationProbabilityGPS{} processes each measurement
and constitutes a core of the measurement domain algorithm.
As a first step it calculates effective clock drift and clock offset for this
measurement as is a sum of the receiver clock drift and offset and of the 
per-SV residual clock drift and offset.
Next a corrected pseudorange and velocity are computed.
Here the expression for calculating the velocity is based on a reported velocity
as used in SiRF receiver, instead of the full calculation from reported frequency.
For pseudoranges the delays introduced in the atmosphere must be compensated,
here represented by the function \delays{}.
Then the expected range and velocity are calculated.
Range computation is trivially the length of the vector between the user and 
satellite positions in ECEF coordinate frame, expected velocity is obtained
as a dot product of unit vector from the user to satellite and velocity distance.
Finally the probabilities are acquired from normal distribution PDF, if they
are within threshold.

\begin{figure}[p]
\begin{algorithm}[H]
    \SetKwFunction{abs}{abs}
    \SetKwFunction{exp}{exp}
    \SetKwFunction{dotProduct}{dotProduct}
    \SetKwFunction{norm}{norm}
    \SetKwFunction{normpdf}{normpdf}

    \SetKwData{samples}{samples}
    \SetKwData{sample}{s}
    \SetKwData{samplex}{sample}


    \SetKwData{deltaT}{deltaT}
    \SetKwData{lastTime}{lastTime}
    \SetKwData{clockDrift}{clockDrift}
    \SetKwData{clockOffset}{clockOffset}
    \SetKwArray{residualDrift}{svDrift}
    \SetKwArray{residualOffset}{svOffset}
    \SetKwData{sv}{sv}
    \SetKwData{svs}{svList}
    \SetKwData{geometricDistance}{geometricDistance}
    \SetKwData{pseudorange}{pseudorange}
    \SetKwData{position}{position}
    \SetKwData{velocity}{velocity}
    \SetKwData{relativeVelocity}{relativeVelocity}
    \SetKwData{userToSv}{userToSv}
    \SetKwData{geomRange}{geomRange}
    \SetKwData{geomVelocity}{geomVelocity}
    \SetKwData{pseudorangeError}{pseudorangeError}
    \SetKwData{velocityError}{velocityError}
    \SetKwData{rangeError}{rangeError}
    \SetKwData{velocityError}{velocityError}
    \SetKwData{observation}{observation}
    \SetKwData{result}{probability}

    \function{predictGPS}{
        \KwIn{list of samples}
        \KwOut{modified list of samples}
        \BlankLine
        \deltaT \assign \timeFunc{} - \lastTime\;
        \lastTime \assign \timeFunc{}\;
        \ForEach{\sample \(\in\) \samples}{
            \sample.\clockDrift \assign \sample.\clockDrift \(+\) \norm{\(\mu\) = \num{\clockDriftRateMu}, \(\sigma\) = \num{\clockDriftRateSigma}} \(*\) \deltaT\;
            \sample.\clockOffset \assign \sample.\clockOffset \(+\) \sample.\clockDrift \(*\) \deltaT\;

            \BlankLine
            \ForEach{\sv \(\in\) \svs}{
                \sample.\residualDrift{\sv} \assign \sample.\residualDrift{\sv} \(+\) \norm{\(\mu\) = \num{\residualClockDriftRateMu}, \(\sigma\) = \num{\residualClockDriftRateSigma}} \(*\) \deltaT\;
                \sample.\residualOffset{\sv} \assign \sample.\residualOffset{\sv} \(+\) \sample.\residualDrift{\sv} \(*\) \deltaT\;
            }
        }
        \Return{\samples}\;
    }

    \BlankLine
    \BlankLine
    \BlankLine
    \function{observationProbabilityGPS}{
        \KwIn{sample, observation}
        \KwOut{probability}

        \BlankLine
        \sv \assign \observation.\sv\;
        \clockDrift = \samplex.\clockDrift + \samplex.\residualDrift{\sv}\;
        \clockOffset = \samplex.\clockOffset + \samplex.\residualOffset{\sv}\;

        \BlankLine
        \pseudorange \assign \observation.\pseudorange \(+\) \(\speedoflight\) \(*\) \sv.\clockOffset \(-\) \(\speedoflight\) \(*\) \clockOffset\\
            \hspace{1em} \(-\) \delays{\sv, \samplex}\;
        \velocity \assign \observation.\velocity \(-\) \(\speedoflight\) \(*\) \clockDrift\;

        \BlankLine
        \userToSv \assign \sv.\position \(-\) \samplex.\position\;
        \relativeVelocity \assign \sv.\velocity \(-\) \samplex.\velocity\;

        \geomRange \assign \abs{\userToSv}\;
        \geomVelocity \assign \dotProduct{\relativeVelocity, \userToSv} / \geomRange\;

        \BlankLine
        \rangeError \assign \pseudorange \(-\) \geomRange\;
        \uIf{\abs{\rangeError \(-\) \pseudorangeErrorMu} \(<\) \num{\pseudorangeThreshold}}{
            \result \assign \normpdf{\rangeError, \(\mu\) = \num{\pseudorangeErrorMu}, \(\sigma\) = \num{\pseudorangeErrorSigma}}\;
        }
        \Else{
            \result \assign \num{\pseudorangeThresholdProbability}
        }
        \velocityError \assign \velocity \(-\) \geomVelocity\;
        \uIf{\abs{\velocityError \(-\) \velocityErrorMu} \(<\) \num{\velocityThreshold}}{
            \result \assign \result \(*\) \normpdf{\velocityError, \(\mu\) = \num{\velocityErrorMu}, \(\sigma\) = \num{\velocityErrorSigma}}\;
        }
        \Else{
            \result \assign \result \(*\) \num{\velocityThresholdProbability}
        }
        \Return \result\;
    }
\end{algorithm}
\caption{Motion and sensor models for measurement domain integration.}
\label{algo:gps-measurement-domain}
\end{figure}

