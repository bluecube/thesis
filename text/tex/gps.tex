\chapter{GPS}
\label{chap:gps}

The Global Positioning System is a global navigation satellite system (GNSS)
developed and operated by the US government.
GPS provides three-dimensional position and velocity measurements as well as
precise time source for every place on Earth with direct satellite visibility.
The GPS is not the only GNSS, in use, but arguably the only one in wide use.
Very brief overview of other navigation systems can be found in \Cref{sec:gps-alternatives}.

This chapter contains an overview of the Global Positioning System and
the basics of its operation.
This will be used to develop methods for using GPS
measurements in Monte Carlo localization in \Cref{chap:gps-and-mcl}.

First, principles of GPS are discussed, followed by a description of methods for
obtaining receiver position and velocity,
error sources of GPS measurements and methods of dealing with these errors.
Lastly this chapter deals with communication with consumer grade GPS receivers
and also briefly with worldwide positioning systems other than GPS.

Unless other sources are cited, this chapter is based on chapters
2 and 7 of \cite{kaplan06}.

The document \cite{fyfe92} contains a specification of the GPS user
segment interface, describing in depth many parts of GPS system.
However since in this work we only interface with the GPS system through a
consumer-grade receiver, we are shielded from much of the low level details of
the GPS.

\section{Basic Description}
GPS consists of a constellation of at least 24 satellites (space vehicles, SVs),
a control segment and end-user receivers.

The satellites are orbiting in six nearly circular orbits approximately \SI{20200}{\kilo\meter}
above Earth surface.
The satellite constellation is still being modernized \cite{gps-modernization-www}
and new satellites are being launched.
Several GPS signals with different properties are available and more will become available
with modernized satellites,
but generally all these signals can be divided to encrypted military signals 
(Precise Positioning Service) and unencrypted
civilian signals (Standard Positioning Service).
The civilian signals are available worldwide without any limitations.


GPS is often viewed as a simple and reliable means of navigation,
but especially the unencrypted signals are susceptible to spoofing attacks \cite{tippenhauer11}.


The control segment consists of a network of ground facilities and is responsible for
monitoring the satellite constellation and for sending commands and data to the satellites.

%\section{History}
%Prior to development of GPS, a satellite navigation system called Transit was operational
%since 1964.
%Transit measured 2D positions of stationary or slow moving receivers based on
%Doppler shifts of the satellite signal \cite{transit-www}.


\section{Operation Principles}

\subsection{Time of Arrival Measurements}

GPS calculates position by measuring the time it takes for a signal
emitted from a known position to reach the receiver.

Each GPS satellite transmits a pseudo random sequence (PRN signal),
containing timing information.
The receiver is able to replicate and match this sequence and therefore
is able to determine the transmission time of the signal.

If the position of satellites is known and all clocks in the system are
synchronized, the position of the receiver
can be calculated as an intersection of at least three spheres centered around the satellites
and with radius corresponding to time of flight of the signal.

In reality, receiver clock offset needs to be calculated
during localization witch adds a new dimension to position determination.

\subsection{Time and GPS}

One of the functions of the GPS system is a dissemination of precise time.
There are three basic time frames appearing in the GPS system.

\subsubsection{GPS System Time}

GPS System Time is a paper time scale based on atomic clock standards in GPS
satellites and on the ground.
This time standard is not directly available neither in SVs or in user receivers.

It is specified by a week number -- number of Saturday/Sunday midnights since week 0
that started on January, 6th 1980 -- and time of week in seconds.

GPS system time is related to UTC.
It is a continuous time scale, not adjusted for leap seconds
and it is required to be within \SI{1}{\micro\second} from UTC modulo \SI{1}{\second}.

\subsubsection{SV Time}
SV Time is a value that satellites transmit in their ranging signals,
obtained from the satellite's atomic clock.
Although the atomic clock standards are highly stable, the offset between SV
Time and GPS System Time may reach up to \SI{1}{\milli\second} (equivalent to
\SI{300}{\kilo\meter} of measurement error).

Values of this offset are calculated by the control segment
and broadcast by the satellites, or downloaded with precise ephemeris
(see \Cref{sec:gps-ephemeris}).
Because of this we can assume that this value is known, although not
with absolute precision.

\subsubsection{Receiver Time}
Receiver Time is a time that is kept by the GPS receiver clock.

Receivers are usually equipped only with simple crystal oscillators
and aren't capable of keeping precise time.
This is solved by adding the clock offset as a fourth dimension of the receiver position.
Having clock offset as a part of the navigation solution makes it easy to calculate
precise time after a fix is obtained by adding the offset to the receiver clock.

\subsection{GPS Signals}
GPS signals are transmitted using CDMA.
Each signal at each satellite is assigned a unique pseudo random sequence, called PRN code and the carrier frequency is
modulated using this sequence.
Additionally navigation messages are also transmitted on this signal (see further).

Legacy GPS satellites work on two frequencies, primary L1 (\SI{1575.42}{\mega\hertz}) and secondary L2 (\SI{1227.6}{\mega\hertz}).
Unencrypted \SI{1}{\mega\hertz} C/A code is transmitted on L1, and encrypted military \SI{10}{\mega\hertz} P(Y) code is transmitted
on both L1 and L2.

One of the main advantages of multiple frequencies is the ability to calculate corrections for ionospheric delays from the received signals.
To use this feature without the encryption keys for the P(Y) codes, codeless techniques have been developed
that use the encrypted stream of P(Y) data and utilize timing of the bit stream.
Semi-codeless techniques further exploit other properties of P(Y) code and its known relationships to C/A codes.

Modernized GPS satellites provide three additional civilian signals:
L2C, L5 and a military signal M. A fourth civilian signal L1C is planned, that will be compatible with
European navigation system Galileo (see \Cref{sec:galileo}).
At the time of writing this text, the signals L2C and L5 are partially supported and L1C is still only
planned \cite{gps-modernization-www}.

\subsection{Navigation Messages}
Navigation messages are transmitted modulated on the ranging signals at \SI{50}{\bit\per\second}.
These messages contain ephemeris and clock corrections for each satellite
and other data, including meta data on the ephemeris values and estimated
ionospheric parameters.

See chapter 3.3 of \cite{rizos99} for a brief overview of navigation messages,
or section 20.3.3 of \cite{fyfe92} for the details.

\subsubsection{PRN Sequences}
As mentioned above, PRN sequences are unique to every signal and every satellite.

C/A codes have a chipping rate -- the frequency of PRN code modulated on top
of the carrier wave -- of \SI{1.023}{\mega\hertz} and P(Y) code has a chipping rate
of \SI{10.23}{\mega\hertz}.

Receivers typically employ a number of specialised hardware correlators
that attempt to duplicate the PRN codes and match them to the received signal.
The result of a successful matching is an identification of the transmitting satellite,
the transmission time \(\svtxtime\) (appearing in pseudorange definition \eqref{eq:pseudorange})
modulo PRN code length and also phase and Doppler shift of the carrier wave.

\subsection{Reference Frames}
\label{sec:ref-frames}

\subsubsection{ECEF}
ECEF -- meaning Earth-centered Earth-fixed -- is a Cartesian reference frame widely used in the GPS system.
As the name suggests, the origin of ECEF reference frame is in the Earth's center of mass, the \(XY\) plane is coincident
with the equatorial plane.
\(X\) axis in the direction of \ang{0} longitude, \(Y\) axis in the direction of latitude
\ang{90} East and \(Z\) axis pointing in
the direction of geographical North pole.
Illustration of this can be seen in \Cref{fig:ecef}.

\begin{figure}[h]
	\centering
	\input{img/ecef.pdf_tex}
	\caption{Diagram of ECEF reference frame.}
	\label{fig:ecef}
\end{figure}

Results of ephemeris calculations are in ECEF coordinates
(see \Cref{sec:gps-ephemeris}) and receiver positions are calculated
in ECEF as well.
All later GPS calculations in this work will be in ECEF reference frame.

\subsubsection{WGS84}
The World Geodetic System 1984, defined in \cite{nima04},
serves as an Earth model for GPS.
Exact values of the model have been updated several times in the past,
but these changes have been too small to be problematic for most practical
applications.

The WGS84 reference ellipsoid is used to convert between ECEF coordinates and
Latitude/Longitude geodetic coordinates.
The exact procedure for this conversion can be found in \cite{nima04}.
Height measured from the ellipsoid can be obtained during the conversion,
however historically heights are measured from sea level.
Sea level roughly corresponds to the geoid -- a surface with constant gravity
potential -- also defined in WGS84.

%with semi major axis \(a = \SI{6378137.0}{\meter}\)
%and inverse flattening \(\frac{1}{f} = \frac{a}{a - b} = 298.257223563\).


\subsection{Ephemeris}
\label{sec:gps-ephemeris}

Ephemeris in the context of GPS means the data describing satellite positions and velocities
at a given time.
Ephemeris data are transmitted in the \SI{50}{\bit\per\second} navigation messages,
together with the clock offset and drift data of the satellites, ionospheric
delay estimates and other data.

Satellite orbits in the ephemeris messages are described using a set of Keplerian orbital parameters.
Since in this work we will be only using ephemeris data in the ECEF reference frame, Keplerian parameter
will not be discussed here.
More details can be for example found in section 3.3.3 of \cite{rizos99}, or in \cite{kaplan06}.

\subsubsection{Precise Ephemeris}
Precise ephemeris are satellite ephemeris data calculated by ground stations
and distributed separately from the GPS broadcast, usually with higher precision
than the data available from the GPS satellites.

One of the sources of precise ephemeris is for example \cite{orbit-data}, providing datasets with
post-processed data and also with predictions for the next several hours.
While the quality of these predictions is lower than of the off-line data,
they are still several times more precise than the broadcast ephemerides
(\SI{5}{\centi\meter} RMS versus \SI{100}{\centi\meter} RMS, according to \cite{orbit-data}).

Precise ephemeris files consist of satellite positions in ECEF
coordinates and clock offsets, both sampled in 15 minutes intervals and must be interpolated
before using them for navigation \cite{schenewerk03}.

\section{Obtaining Position and Velocity}


\subsection{Pseudorange}
\label{sec:pseudorange}

Pseudorange (\(\rho\)) is a distance that corresponds to the time taken by the ranging
signal to travel from SV to the receiver, including the clock offsets:
\begin{equation}
	\label{eq:pseudorange}
	\rho = \speedoflight (\recrxtime - \svtxtime)
\end{equation}
Here \(\speedoflight\) is the speed of light, with value \SI{299792458}{\meter\per\second} used in GPS,
\(\recrxtime\) and \(\svtxtime\) are the receive time referenced to the receiver internal clock
and transmit time according to the satellite clock.
Similarly \(\systxtime\) and \(\sysrxtime\) are the transmit and receive times referenced
to the GPS system time.
Additionally, \(\svclockoffset\) and \(\recclockoffset\) stand for the clock offset of the satellite
and of the receiver, \(\geomrange\) is the real geometric distance between the satellite and receiver
and \(\geomrangedelays\) are the signal propagation delays.

Especially in the context of pseudoranges, GPS literature often treats distance and time
interchangeably, converting between them using the speed of light \(\speedoflight\).
In this work, we will occasionally follow this convention as well.

\begin{figure}[tb]
	\centering
	\input{img/pseudorange.pdf_tex}
	\caption{Times in pseudorange measurements.}
	\label{fig:pseudorange}
\end{figure}

In \eqref{eq:pseudorange} the definition of pseudorange, the SV and receiver clocks can be
converted to system time by subtracting their clock offsets:
\begin{equation}
	\label{eq:pseudorange2}
	\rho = \speedoflight (\sysrxtime - \systxtime) + \speedoflight (\recclockoffset - \svclockoffset)
\end{equation}
It should be noted that the clock offsets change in time, and that \(\recclockoffset\) is receiver clock
offset in the time of receiving and \(\svclockoffset\) applies to the time of transmission.

Real geometric range \(\geomrange\) can be described using the transmit and receive times:
\begin{equation}
	\geomrange + \speedoflight\geomrangedelays = \speedoflight (\sysrxtime - \systxtime)
\end{equation}
where \(\geomrangedelays\) are delays of the signal propagation.
Together with \eqref{eq:pseudorange2}, the previous equation gives us
the basic equation for determining geometric distance from pseudorange measurements:
\begin{equation}
    \label{eq:pseudorange4}
	\geomrange + \speedoflight\geomrangedelays = \rho - \speedoflight (\recclockoffset - \svclockoffset)
\end{equation}

%\begin{equation}
%	\geomrange - \speedoflight \recclockoffset = \rho -
%		\speedoflight \geomrangedelays + \speedoflight \svclockoffset
%\end{equation}

\Cref{fig:pseudorange} summarizes the relations between times in a single pseudorange measurement.

\subsection{Position}
\label{sec:gps-position}

If the position of satellites \(\svpos_i = \coord{x_{{\SV}i}}{y_{{\SV}i}}{z_{{\SV}i}}\) and the satellite
clock offsets \(\svclockoffset_i\) at the time of transmission are known,
calculating receiver position from the pseudorange measurements means solving a set of non-linear equations
\begin{equation}
	\label{eq:pseudorange3}
	\sqrt{(x_\REC - x_{{\SV}i})^2 + (y_\REC - y_{{\SV}i})^2 +(z_\REC - z_{{\SV}i})^2} +
	\speedoflight (\recclockoffset - \svclockoffset_i)
	=
	\rho_i - \speedoflight\geomrangedelays_i
\end{equation}
for receiver position \(\recpos = \coord{x_\REC}{y_\REC}{z_\REC}\) and receiver clock offset \(\recclockoffset\).

This corresponds to an intersection of conical surfaces in four dimensions.
\questiontodo{Draw a picture here?}

Equations from \eqref{eq:pseudorange3} can be solved using various methods, including closed form solutions
(for example \cite{leva96}), iterative solutions, linearization of the equations around
previous position estimates, using Kalman filters (see \Cref{sec:kalman}) or,
as we will discuss in \Cref{sec:measurement-domain}, Monte Carlo localization.
The last two approaches also have the advantage of being able to fuse other sensor data
into the solution.

\subsubsection{Carrier Phase Tracking}
\label{sec:gps-carrier-phase}

Carrier phase tracking is a technique that depends on measuring the phase of carrier wave.
L1 frequency has a wavelength of approximately \SI{19}{\centi\meter}.
Assuming the receiver can match the L1 with \SI{1}{\percent} accuracy, then the available precision
is around \SI{2}{\milli\meter} compared to about \SI{3}{\meter} for code phase (pseudorange)
tracking. More detailed description of the code phase tracking errors is in \Cref{sec:gps-error-sources}.

One of the main problems with the carrier phase tracking is integer ambiguity of the carrier wave.
This problem arises because, unlike the PRN signals, the carrier wave doesn't contain any code designed to
help the receiver distinguish between successive periods of the signal.

Carrier phase measurements can be used to smooth pseudorange data.
Details about this method can be found in section \enquote{Smooth} in \cite{sam-www}.

\subsection{Velocity}
The simplest way of obtaining the receiver velocity is as a derivation of
position, which is obtained by using any of the methods described in the preceding paragraphs.
This has the advantage of requiring only a minimum of additional processing.

When the receiver position is known, its velocity can also be obtained from the Doppler shift of
the received signal and known velocities of satellites.
The received frequency can be approximated as
\begin{equation}
	f = f_\SV \left(1 - \frac{\recsvvel}{\speedoflight}\right)
	\label{eq:doppler}
\end{equation}
where \(f_\SV\) is the transmitted frequency and \(\recsvvel\) is the relative velocity
between the satellite and the receiver.
The relative velocity can be written as a dot product of an unit vector pointing
from user to the satellite and velocity difference between the user and the satellite:
\begin{equation}
	\recsvvel = \frac{
		(\svpos - \recpos)
	}{
		\norm{\svpos - \recpos}
	} \cdot (\svvel - \recvel)
	\label{eq:doppler-velocity}
\end{equation}

When calculating receiver velocities, the measured values also have to be corrected for
the clock drifts, both in satellite and in receiver.
Clock drifts are specified in seconds per second and determine the rate of change of clock offset.
Satellite clock drifts are transmitted together with their clock corrections, so we can ignore
them in a similar fashion as satellite clock offsets, but receiver clock drifts must be determined
together with receiver velocity.

The physically received frequency \(f\) is related to the frequency \(f_\REC\) reported by the receiver
using the receiver clock drift \(\recclockdrift\):
\begin{equation}
	f = f_\REC ( 1 + \recclockdrift)
	\label{eq:doppler-clock-drift}
\end{equation}

By putting \Cref{eq:doppler,eq:doppler-clock-drift}
together, we obtain
\begin{equation}
    \label{eq:doppler2}
	\recsvvel
	=
	\speedoflight \left(1 - \frac{f_\REC}{f_\SV}\right) -
	\speedoflight \frac{f_\REC}{f_\SV} \recclockdrift
\end{equation}

This equation arises for every satellite in view and
as with pseudoranges and positions, there are several ways to obtain the receiver velocity
from them.
In this work we will only use the Doppler measurements as an input to Monte Carlo localization
and we will not discuss the other methods.

\section{Measurement Errors}
\label{sec:gps-errors}

Previous text assumed that all measurements in the GPS system can be made accurately,
but practically all the segments of the GPS system introduce errors.

\subsection{User Equivalent Range Error}
\label{sec:gps-uere}

%\begin{table}[t]
%\centering
%\begin{tabular}{lS}
%\toprule
%	Error Source			& \(1\sigma\) error (\si{\meter}) \\
%\midrule
%	Broadcast clock			& 1.1 \\
%	L1 P(Y)-L1 C/A group delay	& 0.3 \\
%	Broadcast ephemeris		& 0.8 \\
%	Ionospheric delay		& 7.0 \\
%	Tropospheric delay		& 0.1 \\
%	Receiver noise and resolution	& 0.1 \\
%	Multipath			& 0.2 \\
%\midrule
%	Total UERE			& 7.1 \\
%\bottomrule
%\end{tabular}
%\caption{Example GPS error budget.}
%\footnotesize Taken from \cite{kaplan06}
%\label{tab:error-budget}
%\end{table}

User equivalent range error (UERE) characterizes the effective accuracy of a pseudorange measurement.
UERE is defined as a sum of errors caused by different parts of the GPS system.

UERE and its components are usually assumed to be zero mean Gaussian variables,
mutually independent both between the error components of a single measurement and between satellites.

\subsection{Dilution of Precision}
\label{sec:gps-dop}

Dilution of precision (DOP) is a value that describes the effects of the satellite geometry
on the precision of the calculated fix.
There are several types of dilution of precision, specifying errors in different directions.
These are PDOP (position in all directions), HDOP (horizontal error), VDOP (vertical error)
or TDOP (time error).

In theory, standard deviation of the completed fix should have a linear dependency on
UERE and DOP (see equation \eqref{eq:dop-def}), but the actual position error is typically slightly lower,
since the ionospheric errors are correlated.

In \Cref{sec:position-domain} we are discussing dependence of the position error on HDOP value
and \Cref{fig:wgs84-hdop-error} shows the experimental data.

\subsubsection{Definition}
The formal definition of DOP values is based on linearization of pseudorange equations
in \Cref{sec:gps-position}:
\begin{equation}
	\label{eq:dop1}
	\vect{H}\vect{\Delta{}x} = \vect{\Delta{}\rho}
\end{equation}
Here \(\vect{H}\) is a matrix of unit vectors pointing from the linearization point
to the satellites, \(\vect{\Delta{}\rho}\) is a vector containing the difference between
pseudoranges in linearization point and the real measured pseudoranges and finally
\(\vect{\Delta{}x}\) is a vector describing the offset of receiver position from the
linearization point.
To simplify the final definitions of DOP values, all calculations here are performed
in a local reference frame with the \(z\) axis pointing up from the linearization point.

Equation \eqref{eq:dop1} is solved for \(\vect{\Delta{}x}\) using least squares and
as a next step, pseudorange and position errors are taken into account.
This yields an expression relating position error to pseudorange error:
\begin{equation}
	\vect{\delta{}x} = (\vect{H}^T \vect{H})^{-1} \vect{H}^T \vect{\delta{}\rho}
\end{equation}
Where \(\vect{\delta{}x}\) and \(\vect{\delta{}\rho}\) are position errors and pseudorange errors.

As a next step, pseudorange and position errors are taken into account and after
some modifications, relations between position and pseudorange error are expressed.
All pseudorange errors are assumed to be independent, Gaussian and identically distributed.

The covariance matrix of \(\vect{\delta{}x}\) can then be expressed as
\begin{equation}
	\cov(\vect{\delta{}x}) =
		(\vect{H}^T \vect{H})^{-1} \sigma^2_\UERE =
		\begin{pmatrix}
			\sigma^2_{x}  & \sigma^2_{xy} & \sigma^2_{xz} & \sigma^2_{xt} \\
			\sigma^2_{xy} & \sigma^2_{y}  & \sigma^2_{yz} & \sigma^2_{yt} \\
			\sigma^2_{xz} & \sigma^2_{yz} & \sigma^2_{z}  & \sigma^2_{zt} \\
			\sigma^2_{xt} & \sigma^2_{yt} & \sigma^2_{zt} & \sigma^2_{t}
		\end{pmatrix}
\end{equation}
The covariance matrix only depends on satellite geometry as described by \(\vect{H}\) and UERE.

DOP values are defined as follows:
\begin{subequations}
	\label{eq:dop-def}
\begin{align}
	\sqrt{\sigma^2_x + \sigma^2_y + \sigma^2_z} &= \mathrm{PDOP} \sigma_\UERE \\
	\sqrt{\sigma^2_x + \sigma^2_y} &= \HDOP \sigma_\UERE \\
	\sqrt{\sigma^2_z} &= \mathrm{VDOP} \sigma_\UERE \\
	\sqrt{\sigma^2_t} / \speedoflight &= \mathrm{TDOP} \sigma_\UERE
\end{align}
\end{subequations}

\subsubsection{DRMS}
Distance root mean square is an accuracy metric, somewhat similar to
standard deviation of a distribution.
It is a single value describing the accuracy of a 2D position fix.

DRMS is defined as
\begin{equation}
	\mathrm{DRMS} = \sqrt{E(\norm{\vect{\delta{}R}}^2)}
\end{equation}
where \(\vect{\delta{}R}\) is a horizontal component of the position error.

Theoretically DRMS can be obtained from HDOP values:
\begin{equation}
	\mathrm{DRMS} = \HDOP \sigma_\UERE
\end{equation}

\cite{www-wilson} offers an alternative expression for obtaining
DRMS from HDOP:
\begin{equation}
	\mathrm{DRMS} = \sqrt{(a \HDOP)^2 + b^2)}
\end{equation}
where \(a\) and \(b\) are parameters that are fitted to measured data.
In \cref{sec:wgs84-hdop-error} we show how this expression fits our experimental data.

\(2 \times \mathrm{DRMS}\) is often used as an approximation for a radius containing
\SI{95}{\percent} of measurements.

\subsection{Error Sources}
\label{sec:gps-error-sources}

The following text describes the major sources of measurement errors and also a way
of removing or at least limiting them.
Apart from errors mentioned in this section, there are many more sources, but they cause inaccuracies
that are negligible for single receiver applications discussed in this text.
An overview can be found in \cite{kouba09}.

\subsubsection{Match Accuracy and Receiver Delays}

When the receiver matches the received signal to expected PRN codes, an
important question arises -- how accurate the matching is.
Often cited value is \SI{1}{\percent}, which for \SI{1.023}{\mega\hertz}
chipping rate gives about \SI{2.9}{\meter}.
This kind of errors can be improved by using higher quality receiver hardware
or by smoothing with carrier phase measurements (see \cref{sec:gps-carrier-phase}).

Delays also occur in receiver signal processing, both in the electronics and in software,
but if these delays are constant for all pseudorange measurements, they will be removed
together with receiver clock delay.

\subsubsection{Multipath}

\begin{figure}[t]
	\centering
	\input{img/multipath.pdf_tex}
	\caption{Multipath and shadowing.}
	\label{fig:multipath}
\end{figure}

Multipath errors appear when the signal from the satellite reaches the receiver
through multiple paths of different length.
They are typically caused by buildings surrounding the receiver position,
reflective surfaces surrounding the antenna (e.g. wings in an aircraft mounted GPS) or Earth's surface
when including satellites with low elevation angle in the solution.

An illustration of this effect and of shadowing is shown in \Cref{fig:multipath}.
The signal drawn in green follows a direct path from the satellite to the receiver and is shadowed,
and the red multipath signal is reflected from a building, making its travelled path longer.

If the signal following the direct path is received, GPS receivers can easily detect and remove large delays caused by multipath.
A more problematic case arises when the multipath delay is relatively small, because this distorts
receiver's attempts to correlate the received signal with the internal reference signal.

Shadowing is another phenomenon related to multipath, caused by the signal passing through obstacles, like buildings or foliage.
Shadowing combined with multipath affects the relative received power of direct and multipath signals, in extreme cases
even causing the multipath to be the only received signal from a given satellite.

Multipath errors can be mitigated with hardware modifications, like antenna designs that suppress signals from suspicious angles,
better antenna placement or coating nearby reflective surfaces with RF absorptive materials.
Another option is to employ better signal processing in receivers.

When only complete measurements are available (as is the case in the implementation part of this text),
satellites may be dropped from the solution if the elevation angle
is too low to avoid this kind of errors.
A more complex method of dealing with multipath errors is described in \cite{viandier08},
where the error model of a satellite measurement is modified if it is suspected to be corrupted
by multipath error.

Section 6.3 of \cite{kaplan06} contains a detailed discussion of multipath effects.

\subsubsection{Tropospheric Delays}
Tropospheric delays describe signal delays in the non-ionized layers of Earth's atmosphere.

Troposphere is the non-ionized layer of atmosphere closest to the surface of the Earth.
The stratosphere -- another non-ionized layer of atmosphere -- causes the same
type of delays and is usually included in the therm tropospheric delay.

Tropospheric delays are modelled as consisting of a wet and a dry component,
and when left uncompensated, they amount to between \SI{2.4}{\meter} and \SI{25}{\meter}
depending on satellite elevation angle, the dry component
responsible for about \SI{90}{\percent} of the delay \cite{kaplan06}.

There are several models for estimating tropospheric delays depending on satellite
elevation angle, altitude of the receiver, atmospheric pressure, temperature and
water vapour pressure.

In the implementation part of this work we use the model outlined in \cite{navipedia-tropospheric}
with hard-coded meteorologic parameters.

\subsubsection{Ionospheric Delays}

Compared to tropospheric delays, signal delays caused in the ionosphere are larger and
harder to predict.

An interesting property of ionospheric delays is, that they are correlated,
both in measurements from different satellite to a single receiver and in measurements
from a single satellite to multiple receivers.
The later property is utilized in differential GPS (\Cref{sec:dgps}).

Since the ionospheric delay is a function of signal frequency and total electron
count in atmosphere \cite{kaplan06}, it can be almost completely
eliminated when using multi frequency receivers.
Single frequency receivers, are however forced to use less accurate
models of the ionosphere and ionospheric parameters (Klobuchar model for GPS
\cite{navipedia-klobuchar} or NeQuick model for Galileo \cite{navipedia-nequick}).
Ionospheric correction estimates are also transmitted in the GPS navigation
messages \cite{fyfe92}.

\subsubsection{Ephemeris and SV Clock Errors}

Ephemeris and satellite clock corrections are periodically uploaded by the control
segment, however the residual errors increase with the age of upload.
According to \cite{kaplan06}, \(1 \sigma\) pseudorange error is typically about \SI{0.8}{\meter}
and up to \SI{4}{\meter} for clock errors.

As discussed in \Cref{sec:gps-ephemeris}, this class of errors can be avoided or at least decreased by using precise ephemeris
at the expense of requiring internet connection during operation.

\subsubsection{Relativistic Effects}

%\paragraph{Satellite Clock Frequency}
The most famous effect of relativity is the frequency shift of the satellite clock
when received on the ground.
This frequency shift happens because of the satellites velocity relative to the user
and also because satellite orbits are further from Earth's mass and therefore less
affected by the spacetime curvature caused by it.

To compensate, satellite clock frequency is set to
\SI{10.22999999543}{\mega\hertz} before launch, so the observed frequency
is \SI{10.23}{\mega\hertz} \cite{fyfe92}.
\todo{Check if it is necessary, maybe expand}

\section{Kalman Filters}
Kalman filters are often used in the GPS receivers to merge past
data with incoming measurements or to combine the GPS position and velocity
estimation with another source of measurements, for example inertial data.

For more detailed description of how Kalman filters operate see \Cref{sec:kalman}.

\section{Differential GPS}
\label{sec:dgps}

Differential GPS is a method to improve GPS accuracy using at least one reference station.
DGPS exploits the correlations of GPS errors and is capable of removing or decreasing satellite clock errors,
ephemeris errors, tropospheric and ionospheric errors.

DGPS systems may operate with baseline distances from hundreds of meters to several thousand kilometers.
They may be used to calculate precise absolute positions or positions relative to the reference station.
Accuracy of DGPS may range from several decimeters for code based systems to several
millimeters for carrier based systems.

Reference stations must feed their data to the receivers, which correct their own measurements base on them.
This need of infrastructure and reliable wireless connection makes DGPS impractical for many uses.

%\cite{dana99}

Differential GPS is often used in surveying (off-line, high precision) and this
use is discussed in \cite{rizos99}.

\section{Assisted GPS}
Assisted GPS refers to the techniques used typically on PDAs and cell phones, that improve operation of GPS by offloading
work from the receiver chip to a remote assistance server.

The assistance server has a good satellite reception and more computing power than the device
and can supply the client system with complete position fixes based on uploaded GPS signal sample,
ephemeris data which can then be used to speed up cold start of the client device or
ionospheric parameters, forming simple DGPS system and increasing the fix precision.

%\section{GPS Receivers}

%As stated before, this work focuses on low cost consumer grade receivers.
%There are currently many different receiver chips available, differing in

\section{Protocols}
Most of the time a consumer GPS receiver is used as a black box that provides position
estimates when power and possibly external antenna is connected.
In more complicated use cases, such as measurement domain integration of GPS signals with
other sensors discussed in \Cref{sec:measurement-domain} of this work,
lower level data from the location estimation process may be used, but only as provided
by software in the GPS receiver.
Many communication protocols are in use, which providing position, velocity, clock data
and other information.

\subsection{NMEA 0183}
NMEA 0183 is a de-facto standard protocol for consumer grade GPS receivers.
It is a text-based proprietary format developed by National Marine Electronics Organization,
but has been reverse engineered and the specification is known and widely used \cite{depriest}.

The protocol by itself is designed for communication of marine electronic
equipment and isn't specific to GPS receivers.
It consists of a large number of sentences, some of which are intended to be used
for position, velocity and time solutions of GPS navigation.

The problem with the NMEA protocol is, that a different subset is implemented in almost every
receiver and there is no access to lower levels of the GPS signal processing.

\subsection{Proprietary Protocols}
Many GPS receivers extend the NMEA protocol using proprietary sentences or
implement custom protocols.
An example of this is the SiRF binary protocol \cite{sirf-protocol}.

The use of proprietary protocol is often the only way to get the complete functionality from
the receiver.

\subsection{RINEX}

The Receiver Independent Exchange Format \cite{rinex-format} is an exchange format
for GPS and other satellite navigation systems.

RINEX is typically used for post processing as it can contain various data that are not
known during navigation, for example detailed ionospheric models or high precision ephemeris.

\section{Similar Systems}
\label{sec:gps-alternatives}

GPS is not the only GNSS currently operational.
Other countries than USA have also created navigation systems, and several regional satellite
navigation systems also exist.
Section 1.5 of \cite{david} contains a more detailed overview.

GLONASS \cite{glonass} is a Russian navigation system, in principle similar to GPS.
It is currently undergoing a modernization and is planned to be compatible with GPS and Galileo.

Another example of a current GNSS system is the Chinese CNSS / COMPASS \cite{compass},
also known as  BeiDou-II.
The system is planned to be operational by 2020.

\subsection{Galileo}
\label{sec:galileo}

Galileo \cite{galileo} is a project by the European Union to create a global navigation system, which
is probably of the most interest to current GPS users.

It is designed to be independent of GPS and to offer multiple levels of access worldwide
including an open access navigation, however
it will be compatible with GPS receivers that support the modernized L1C signals.
The Galileo constellation will consist of 30 satellites, and will provide several types of services,
from free public service to safety-critical services.

Technically, Galileo is fairly similar to GPS.
Their compatibility is provided by using the same geodetic and time reference frames
and as mentioned above, by using a signal compatible with the GPS L1C.
This means, that a GPS receiver supporting this signal will be able to seamlessly
use both systems combined.

Moreover, measurements from Galileo satellites will suffer from the same errors as
their GPS counterparts, which makes this system interchangeable with the GPS for the purposes
of this work.
