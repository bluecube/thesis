\chapter{GPS}
\label{chap:gps}

This chapter contains a overview of Global Positioning System and
its basic operation.
Next it describes methods of obtaining receiver position and velocity,
error sources of GPS measurements and methods of dealing with these errors.
Lastly it discusses properties of consumer grade GPS receivers.

Global Positioning System is a global navigation satellite system (GNSS)
developed and operated by U.S. military.
GPS provides three dimensional position and velocity measurements as well as
precise time source for every place on earth with direct satellite visibility.

GPS consists of a constellation of at least 24 satellites (space vehicles, SVs),
control segment and end-user receivers.

\begin{itemize}
\item The satellites are orbiting approximately \SI{20200}{km} above earth surface.
\item 6 nearly circular orbits
\item Ongoing modernization

\item Control segment consists of a network of ground facilities and is responsible for
monitoring the satellite constellation and send commands and data to the satellites,

\item There are several GPS signals with different properties
and more will become available with modernized satellites,
however all these signals can be divided to encrypted military signals 
(Precise Positioning Service) and unencrypted
civilian signals (Standard Positioning Service).

The unencrypted signals are available worldwide without any limitations,
however they are also susceptible to spoofing attacks, however complicated these are
\cite{tippenhauer11}.

\end{itemize}

%\section{History}
%Prior to development of GPS, a satellite navigation system called Transit was operational
%since 1964.
%Transit measured 2D positions of stationary or slow moving receivers based on
%Doppler shifts of the satellite signal \cite{transit-www}.


%
\section{Operation Principles}
%
\subsection{Time of Arival measurements}

%
%GPS calculates position by measuring the time it takes for a signal
%emitted from a known position to reach reach the receiver.
%
%Each GPS satellite transmits a timed pseudo random sequence.
%The receiver is able to replicate and match this sequence, and therefore
%is able to determine the transmission time of the signal.
%
%Because the position of satellites is known, the position of the receiver
%can be calculated as an intersection of spheres centered around the satellites
%and with radius corresponding to signal time of flight.
%
%The previous paragraph assumes, that clock in the satellites and in the receiver 
%are perfectly synchronized, in reality, however, this is not true.
%
\subsection{Time and GPS}
\begin{itemize}
\item GPS System Time
\item SV Time
\item Receiver Time
\end{itemize}
%There are three basic time frames appearing in the GPS system.
%
%\begin{itemize}
%\item
%GPS System Time
%GPS System Time is a paper time scale based on atomic clock standards in GPS
%satellites and on the ground. This time standard is not directly available
%neither in SVs or in user receivers.
%
%\item
%SV Time
%SV Time is a value obtained from the satellite's atomic clock.
%Although the atomic clock standards are highly stable, the offset between SV
%Time and GPS System Time may reach up to \(1 ms\) (~ \(300 km\)).
%Corrections for this offset are periodically calculated by the control segment
%and uploaded to the satellites for broadcast (and therefore known to the
%receiver).
%
%\item
%Receiver Time
%Receiver Time is a time reading in the GPS receiver.
%Low cost receivers are usually equipped only with simple crystal oscillators
%and aren't capable of keeping precise time.
%

\subsection{PRN signals}
\begin{itemize}
\item How it works
\item Codes: C/A, P(Y), M
\end{itemize}

\subsection{Ephemeris}

\subsection{Reference Frames}
\begin{itemize}
\item ECEF, ECI, WGS84
\end{itemize}

\section{Obtaining Position and Velocity}

\subsection{Pseudorange}

%Pseudorange is a distance corresponding to 
%
\[\rho = c (receiveTime_{receiver} - transmitTime_{SV})\]
%

\section{Position}
\begin{itemize}
\item Some math goes in here
\end{itemize}

\subsection{Carrier phase}
\begin{itemize}
\item Not much to say in here
\end{itemize}

\section{Velocity}
\begin{itemize}
\item More math
\end{itemize}

\section{Measurement errors}
\begin{itemize}
\item Describe UERE
\item describe DOP
\label{sec:gps-dop}
\end{itemize}

\begin{itemize}
\item Multipath
\item Ionospheric \& tropospheric delays
\item Ephemeris errors (+ precise ephemeris web sources)
\item SV clock errors
\end{itemize}

\subsection{Dilution of Precision}
\begin{itemize}
\item effect of satellite geometry on position error
\end{itemize}

\section{Kalman Filters}
Kalman filters are often used in the GPS receivers to merge past
data with incoming measurements or to combine the GPS position and velocity
estimation with another source of measurements, for example acceleration data.

For more detailed description of how Kalman filters operate see section
\ref{sec:kalman}.

\section{Differential GPS}
\begin{itemize}
\item Very briefly; won't be used anywhere further
\item SBAS ?
\end{itemize}

\section{Assisted GPS}

\section{Precise Ephemeris}
Precise ephemeris are satellite ephemeris data calculated by ground distributed
separately from the GPS broadcast \cite{macdonald01}.

One source of precise ephemeris is is \cite{orbit-data}.
These datasets mainly contains post-processed past data, but predictions for
next several hours are also available.
While the quality of these predictions is lower than of the off-line data,
they are still several times more precise than the broadcast ephemerides
(\(5 cm\) RMS versus \(100 cm\) RMS, according to \cite{orbit-data}).


\section{GPS Receivers}
\subsection{Protocols}
\begin{itemize}
\item NMEA
\item Proprietary protocols (especially SiRF)
\end{itemize}

\section{Similar Systems}
\begin{itemize}
\item Galileo
\item Glonass
\item BeiDou
\item Compass
\end{itemize}
