\chapter{GPS}

This chapter contains a description of Global Positioning System,
its history and basic operation.

Global Positioning System is ...

GPS consists of three segments: space segment, control segment and user segment.

Space segment VERB! at least 24 satellites (space vehicles, SVs), orbiting
approximately \(20000 km\) above earth surface.

\section{History}

\section{Basic principles}

\subsection{Time of Arival measurements}

GPS calculates position by measuring the time it takes for a signal
emitted from a known position to reach reach the receiver.

Each GPS satellite transmits a timed pseudo random sequence.
The receiver is able to replicate and match this sequence, and therefore
is able to determine the transmission time of the signal.

Because the position of satellites is known, the position of the receiver
can be calculated as an intersection of spheres centered around the satellites
and with radius corresponding to signal time of flight.

The previous paragraph assumes, that clock in the satellites and in the receiver 
are perfectly synchronized, in reality, however, this is not true.

There are three basic time frames appearing in the GPS system.

\begin{itemize}
\item
GPS System Time
GPS System Time is a paper time scale based on atomic clock standards in GPS
satellites and on the ground. This time standard is not directly available
neither in SVs or in user receivers.

\item
SV Time
SV Time is a value obtained from the satellite's atomic clock.
Although the atomic clock standards are highly stable, the offset between SV
Time and GPS System Time may reach up to \(1 ms\) (~ \(300 km\)).
Corrections for this offset are periodically calculated by the control segment
and uploaded to the satellites for broadcast (and therefore known to the
receiver).

\item
Receiver Time
Receiver Time is a time reading in the GPS receiver.
Low cost receivers are usually equipped only with simple crystal oscillators
and aren't capable of keeping precise time.

\subsubsection{Pseudorange}
Pseudorange is a distance corresponding to 

\[\rho = c (receiveTime_{receiver} - transmitTime_{SV})\]





\end{itemize}
