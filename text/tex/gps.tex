\chapter{GPS}
\label{chap:gps}

%This chapter contains a description of Global Positioning System,
%its history and basic operation.
%
%Global Positioning System is ...
%
%GPS consists of three segments: space segment, control segment and user segment.
%
%Space segment VERB! at least 24 satellites (space vehicles, SVs), orbiting
%approximately \(20000 km\) above earth surface.
%
\section{History}
%
\section{Operation Principles}
%
\subsection{Time of Arival measurements}

%
%GPS calculates position by measuring the time it takes for a signal
%emitted from a known position to reach reach the receiver.
%
%Each GPS satellite transmits a timed pseudo random sequence.
%The receiver is able to replicate and match this sequence, and therefore
%is able to determine the transmission time of the signal.
%
%Because the position of satellites is known, the position of the receiver
%can be calculated as an intersection of spheres centered around the satellites
%and with radius corresponding to signal time of flight.
%
%The previous paragraph assumes, that clock in the satellites and in the receiver 
%are perfectly synchronized, in reality, however, this is not true.
%
\subsection{Time and GPS}
\begin{itemize}
\item GPS System Time
\item SV Time
\item Receiver Time
\end{itemize}
%There are three basic time frames appearing in the GPS system.
%
%\begin{itemize}
%\item
%GPS System Time
%GPS System Time is a paper time scale based on atomic clock standards in GPS
%satellites and on the ground. This time standard is not directly available
%neither in SVs or in user receivers.
%
%\item
%SV Time
%SV Time is a value obtained from the satellite's atomic clock.
%Although the atomic clock standards are highly stable, the offset between SV
%Time and GPS System Time may reach up to \(1 ms\) (~ \(300 km\)).
%Corrections for this offset are periodically calculated by the control segment
%and uploaded to the satellites for broadcast (and therefore known to the
%receiver).
%
%\item
%Receiver Time
%Receiver Time is a time reading in the GPS receiver.
%Low cost receivers are usually equipped only with simple crystal oscillators
%and aren't capable of keeping precise time.
%

\subsection{PRN signals}
\begin{itemize}
\item How it works
\item Codes: C/A, P(Y), M
\end{itemize}

\subsection{Ephemeris}

\subsection{Reference Frames}
\begin{itemize}
\item ECEF, ECI, WGS84
\end{itemize}

\section{Obtaining Position and Speed}

\subsection{Pseudorange}

%Pseudorange is a distance corresponding to 
%
\[\rho = c (receiveTime_{receiver} - transmitTime_{SV})\]
%

\section{Position}
\begin{itemize}
\item Some math goes in here
\end{itemize}

\subsection{Carrier phase}
\begin{itemize}
\item Not much to say in here
\end{itemize}

\section{Velocity}
\begin{itemize}
\item More math
\end{itemize}

\section{Kalman Filters}
\begin{itemize}
\item In end-user receivers.
\end{itemize}

\section{Measurement errors}
\begin{itemize}
\item Multipath
\item Ionospheric \& tropospheric delays
\item Ephemeris errors (+ precise ephemeris web sources)
\item SV clock errors
\end{itemize}

\subsection{Dilution of Precision}
\begin{itemize}
\item effect of sattelite geometry on position error
\end{itemize}

\section{Differential GPS}
\begin{itemize}
\item Very briefly; won't be used anywhere further
\end{itemize}

\section{Assisted GPS}

\section{GPS Receivers}
\subsection{Protocols}
\begin{itemize}
\item NMEA
\item Proprietary protocols (especially SiRF)
\end{itemize}


