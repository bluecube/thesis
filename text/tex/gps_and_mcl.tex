\chapter{GPS and Monte Carlo Localization}
\label{chap:gps_and_mcl}

This chapter discusses two approaches to using GPS receiver as a correction input
to Monte Carlo localization. One based on pseudorange measurements and one based on
WGS84 pre-calculated data.



\section{Position Domain Integration}
\begin{itemize}
\item Using complete fixes from receiver
\item Simple, less CPU intensive
\end{itemize}

\subsection{Error distribution}
\begin{itemize}
\item 2D normal distribution, based on HDOP and "baseline" receiver precision
\end{itemize}

\section{Measurement Domain Integration}
\label{sec:measurement_domain}
\begin{itemize}
\item Using each pseudorange measurement
\item Maybe better localization, no assumptions about vehicle properties
\end{itemize}

\subsection{Static Environment Assumption}
\label{sec:gps-mcl-static-env}
Monte Carlo Localization assumes that the environment is static, but we need to
localize according to moving satellites.
To avoid this problem we will hide satellite motion into the robot state and sensor model.

\subsection{Error distribution}
\begin{itemize}
\item Normal distribution (???)
\end{itemize}

\subsection{Preprocessing}
\begin{itemize}
\item ???
\end{itemize}

\subsection{Initial estimate}
\begin{itemize}
\item starting as position domain, later switching to measurement domain
\end{itemize}

\subsection{Simplifications}
\begin{itemize}
\item Ignoring altitude (???)
\item Additional 1D kalman for altitude (???)
\item Additional 1D kalman for receiver clock error (???)
\end{itemize}
