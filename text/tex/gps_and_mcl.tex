\chapter{GPS and Monte Carlo Localization}

\begin{itemize}
\item Two approaches, both use GPS as a correction input
\end{itemize}

\section{Position Domain Integration}
\begin{itemize}
\item Using complete fixes from receiver
\item Simple, less CPU intensive
\end{itemize}

\subsection{Error distribution}
\begin{itemize}
\item 2D normal distribution, based on HDOP and "baseline" receiver precision
\end{itemize}


\section{Measurement Domain Integration}
\begin{itemize}
\item Using each pseudorange measurement
\item Maybe better localization, no assumptions about vehicle properties
\end{itemize}

\subsection{Error distribution}
\begin{itemize}
\item Normal distribution (???)
\end{itemize}

\subsection{Preprocessing}
\begin{itemize}
\item ???
\end{itemize}

\subsection{Initial estimate}
\begin{itemize}
\item starting as position domain, later switching to measurement domain
\end{itemize}

\subsection{Simplifications}
\begin{itemize}
\item Ignoring altitude (???)
\item Additional 1D kalman for altitude (???)
\item Additional 1D kalman for receiver clock error (???)
\end{itemize}

\section{Implementation}

\subsection{What's specific to SiRF}
\begin{itemize}
\item All measurements normalized to common time
\item Relativistic errors are already taken care of
\item SV positions given in ECEF, instead of orbit parameters
\item \ldots
\end{itemize}

\subsection{Usage}
\begin{itemize}
\item Briefly describe the tools and what they do, how to launch them
\end{itemize}

\section{Performance Comparison}
