\chapter{Implementation}
\label{chap:implementation}

In this chapter describes my implementation of GPS input to Monte Carlo localization,
specifics of the GPS receiver used, design and how to use the implementation.

\section{What's specific to SiRF}
\begin{itemize}
\item All measurements normalized to common time
\item Relativistic errors are already taken care of
\item SV positions given in ECEF, instead of orbit parameters
\item \ldots
\end{itemize}

\section{Design}
\begin{itemize}
\item Record / replay
\end{itemize}

\subsection{Communication Wiht SiRF Chip}
\begin{itemize}
\item Detection / switching: SiRF binary <-> NMEA
\end{itemize}

\subsection{Obtaining the error model}
\begin{itemize}
\item Detection / switching: SiRF binary <-> NMEA
\end{itemize}

\subsection{Monte Carlo Localization Input}
\begin{itemize}
\item ???
\end{itemize}

\section{Usage}
\subsection{Package \lstinline=gps=}
Python package \lstinline=gps= is a tool for communication with SiRF GPS receivers.
It connects to the gps receiver and switches it between NMEA and binary SiRF protocol.
Also the package provides higher level access to the already parsed SiRF messages.

Recorded data stream may also be opened transparently.

\subsection{average\_position.py}
\verb=average_position.py= is a script that calculates a mean position from recorded
GPS data of a static receiver.
It takes complete fix estimation from \verb=MeasureNavigationDataOut= message and simply
calculates a mean of its \(x\), \(y\) and \(z\) coordinates.

This pre-calculated value can later be used to speed up UERE calculations.

\subsection{checksum.py}
\verb=checksum.py= calculates CRC32 checksum of SiRF binary data to verify
correctness of record files.

\subsection{record.py}
The script \verb=record.py= records stores SiRF binary messages from a GPS (or from a recording).
The recording can then be opened by the \lstinline=gps= package in the same way as a real GPS
device.

\subsection{old\_sv\_state.py}
Calculates position errors obtained by using older SV state received from
the GPS rather than the latest value.

Given a recording it prints mean position error.
