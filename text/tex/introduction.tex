\chapter{Introduction}

% Autonomous outdoor robots are fun, although kinda hard.

The ability to determine its position is important for autonomous or semi-autonomous robots.
Not only for completing the robot's task, but also for a basic
safe navigation though the environment.

Outdoor robots may navigate large areas without detailed maps available, thus limits
the usefulness of some sensors used for indoor localization, like laser range finders.
This leaves a robot designer looking for alternative sensors.

Global positioning system (GPS) is a satellite navigation system that
provides position, velocity and time information for receivers at any place on Earth with direct
satellite visibility.

Seeing its use in automotive industry, GPS seems to be an good candidate for
localization of robots operating in outdoor areas.
The best position measurement precisions available from the Global Positioning System
are in the orders of millimeters with survey grade hardware.
Consumer grade receivers, however, only offer precision of several meters, which
by itself isn't good enough for the robot to navigate safely.


Monte Carlo localization (MCL) is an algorithm that estimates position of a robot
based on noisy measurements, possibly from multiple sensors.
Monte Carlo localization is a type of Markov localization that represents the position estimation as
a set of samples.
This estimate is periodically updated in prediction and correction steps based on
actions the robot on measured sensor data.


\vspace{1.5em}

Goal of this work is to acquire as much precision as possible from a single low cost
GPS receiver by using it as an input to MCL and to make it possible to use
GPS as one of the primary sensors for robot localization.

A relatively straightforward way of integrating the GPS data to any localization algorithm is
to use WGS84 geodetic coordinates from the standard NMEA protocol.
With this approach, however, Kalman filters, position smoothing and other filters typically
employed in  consumer grade receivers assume that the receiver is mounted in a car or carried on foot
and modify the data based on these assumptions.
Examples of these assumptions include minimum speed threshold or vertical speed limits.
Additionally, based on NMEA data, measurement errors can only be characterized very roughly.

A different approach that will be explored in this work, uses raw
pseudorange measurements from the GPS receiver.
This is similar in principle to how IR/ultrasound ranging beacons are used.
\questiontodo{some reference?}
When using pseudorange measurements, localization algorithm can work with motion model closely
matching the real hardware, each pseudorange measurement can have an
independent error model and further modifications to improve precision like precise
ephemeris can be used.


\vspace{1.5em}


This work shows the two mentioned  methods of integrating GPS measurements
into Monte Carlo Localization and compares their performance on a real robot.
Also an overview of GPS system and Monte Carlo localization is included in
this an thesis to provide context for the main topic.


The rest of the text is organized as follows:
\Cref{chap:gps} describes the GPS system and its operating principles.
\Cref{chap:mcl} describes Monte Carlo Localization.
\Cref{chap:gps-and-mcl} proposes fusing of GPS data into MCL.
\Cref{chap:implementation} describes implementation of experiments with GPS and
Monte Carlo localization,
and \cref{chap:performance} compares performance of the implemented methods.
\Cref{chap:conclusion} concludes and evaluates the work.
