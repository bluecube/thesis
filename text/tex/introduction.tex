\chapter{Introduction}

%Localization is an important task for autonomous or semi-autonomous robots

Outdoor robots may navigate large areas without detailed maps available, this limits
the usefulness of some sensors used for indoor localization, like laser range finders.
This leaves a robot designer looking for alternative sensors.




Global positioning system, or GPS for short, is a satellite navigation system that
provides position, velocity and time information for any place on Earth with direct
satellite visibility.

Seeing its use in automotive industry, GPS seems to be an good candidate for
localization in robots operating in outdoor areas.
However if GPS was to be used as one of the primary methods for robot localization,
precision offered by consumer grade receivers isn't enough to let the robot navigate safely.

The best position measurement precisions available with GPS are around a millimeter,
however this requires expensive survey hardware.
In this work I will concentrate on low cost consumer grade receivers, which mostly
offer measurement errors in the range of several meters.




Monte Carlo localization (MCL) is an algorithm that estimates position of a robot
based on noisy measurements.
MCL is a type of Bayesian filter that represents the position estimation as a
set of weighted samples.
This estimate is periodically updated and corrected based on sensor input.




The goal of this work is to gain as much precision as possible from low cost
GPS receivers, by using it as an correction input to MCL.



The reset of this text is organized as follows:
Chapter \ref{chap:gps} describes the GPS system and its operating principles.
Chapter \ref{chap:mcl} describes Monte Carlo Localization.
Chapter \ref{chap:gps_and_mcl} chapter talks about fusing of GPS data into
MCL localization.
Chapter \ref{chap:implementation} describes my implementation,
and chapter \ref{chap:performance} compares performance of the implemented algorithms.
Chapter \ref{chap:conclusion} concludes and evaluates the work.
