\chapter{Introduction}

Localization is an important task for autonomous or semi-autonomous robots.
Outdoor robots may navigate large areas without detailed maps available, thus limits
the usefulness of some sensors used for indoor localization, like laser range finders.
This leaves a robot designer looking for alternative sensors.


Global positioning system (GPS) is a satellite navigation system that
provides position, velocity and time information for receivers at any place on Earth with direct
satellite visibility.

Seeing its use in automotive industry, GPS seems to be an good candidate for
localization of robots operating in outdoor areas.
The best position measurement precisions available from the Global Positioning System
are in the orders of millimeters with survey grade hardware.
However consumer grade receivers only offer precision of several meters, which
by itself isn't good enough for the robot to navigate safely.


Monte Carlo localization (MCL) is an algorithm that estimates position of a robot
based on noisy measurements, possibly from multiple sensors.
MCL is a type of Bayesian filter that represents the position estimation as a
set of weighted samples.
This estimate is periodically updated and corrected based on sensor input.


\vspace{1.5em}


The goal of this work is to acquire as much precision as possible from low cost
GPS receivers by using it as an correction input to MCL and to make it possible to use
GPS as one of the primary sensors for robot localization.

I will show two methods of integrating GPS measurements into Monte Carlo Localization,
one of them based on low level GPS measurements and one based on easily accessed WGS84 data
and compare their performance on a real robot.



The reset of this text is organized as follows:
Chapter \ref{chap:gps} describes the GPS system and its operating principles.
Chapter \ref{chap:mcl} describes Monte Carlo Localization.
Chapter \ref{chap:gps_and_mcl} proposes fusing of GPS data into MCL localization.
Chapter \ref{chap:implementation} describes my implementation,
and chapter \ref{chap:performance} compares performance of the implemented algorithms.
Chapter \ref{chap:conclusion} concludes and evaluates the work.
