\chapter{Monte Carlo Localization}
\label{chap:mcl}

The following chapter defines the localization problem, provides quick overview
of Markov localization methods and finally describes Monte Carlo Localization
algorithm.

\section{Localization}
Localization of a mobile robot is a process that estimates position of the robot
in a map based on sensor readings.

Localization can mean either position tracking, where the robot knows its initial
position and the task is only to update the estimate and handle relatively small
errors, or global localization, where the robot has to find its position from scratch.

\section{Markov Localization}

Markov localization is a probabilistic algorithm that estimates state of the
robot based on sensor data.

Markov localization assumes that the environment in which the robot moves
doesn't contain any state (this is called the Markov assumption).
This assumption means that only the last robot's state is necessary to predict
future state.

Markov localization periodically predicts next state of the robot and then
corrects the estimation using the sensor data.

Several variations of Markov localization exists differing in the
representation of the current state estimation:

\subsection{Kalman Filters}
\label{sec:kalman}

Kalman Filters \cite{kalman60,welch95} are maybe the most commonly used variation of Markov localization.

Current estimation is represented as a Multivariate Gaussian distribution.
Basic Kalman Filters assume that the underlying system is linear and that
all error terms have a Gaussian distribution.

Modifications, like the Extended Kalman Filter, are capable of dealing with
non-linear systems.

\subsection{Grids}
Another way of representing the state estimation is a grid.

\subsection{Particle Filters}
Particle Filters represent the estimation by keeping samples from the
probablity distribution.
Monte Carlo Localization is a particle filter.

\section{MCL Algorithm}

\cite{dellaert99}

\subsection{Prediction}
\subsection{Correction}
\subsection{Resampling}
