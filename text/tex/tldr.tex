\chapter{tl;dr}
This chapter provides an overview of the thesis in a form as condensed as possible.
Intended for anyone who knows both GPS and MCL and dislikes reading long texts.

\begin{compactitem}
\item
Using low level GPS data as a sequence of correction inputs to MCL.

\item
Communication with GPS receiver
\begin{compactitem}
    \item Standard NMEA protocol doesn't contain enough detail for advanced usage.
    \item Support for \sirf chips with binary SiRF protocol is implemented, but the
    approach should be adaptable to other receivers without major modifications.
    \item \Cref{sec:impl-sirf}.
\end{compactitem}

\item
NMEA data as MCL input
\begin{compactitem}
    \item Simpler, less CPU intensive, drops some information when the results are
    wrapped in lat / lon / hdop for WGS84.
    \item Working in 2D, no need to modify MCL samples.
    \item
        Fitting curve to measured data:
        \begin{equation*}
            \mathrm{DRMS}(\HDOP) = \sqrt{(\num{4.941}\HDOP)^2 + \num{3.568}^2}
        \end{equation*}
    \item Sensor model based on Rayleigh distribution.
    \item PDF:
        \begin{equation*}
            f(x) = \frac{x}{\sigma^2}e^{-x^2/2\sigma^2} =
            \frac{2x}{\mathrm{DRMS}(\HDOP)^2}e^{-x^2/\mathrm{DRMS}(\HDOP)^2}
        \end{equation*}
    \item Approximately \SI{8}{\meter} DRMS error.
    \item \Cref{fig:wgs84-hdop-error,sec:wgs84-hdop-error,algo:gps-position-domain}.
\end{compactitem}

\item
Pseudoranges as MCL input
\begin{compactitem}
    \item
    Experimental, more complex, higher CPU usage.

    \item
    Working in ECEF coordinates, tracking clock offsets and residual errors.

    \item
    Needs modifications to MCL sample.

    \item
    Residual errors:
    \begin{compactitem}
        \item Per satellite.
        \item Characterized as residual clock drift and residual clock offset
              during processing individual measurements.
        \item Residual clock drift and offset are added to receiver clock drift and offset.
        \item Covers ionospheric and other low-frequency errors.
    \end{compactitem}

    \item
    Motion model:
    \begin{compactitem}
        \item Updating clock drifts, clock offsets, residual drifts.
        \item Clock drifts modelled as Gaussian random walk.
        \begin{compactitem}
            \item Receiver clock drifts:
                  \(\normalDistribution(\mu = \SI{\clockDriftRateMu}{\meter\per\second\squared}, \sigma = \SI{\clockDriftRateSigma}{\meter\per\second\squared})\)
            \item Residual clock drifts:
                  \(\normalDistribution(\mu = \SI{\residualClockDriftRateMu}{\meter\per\second\squared}, \sigma = \SI{\residualClockDriftRateSigma}{\meter\per\second\squared})\)
        \end{compactitem}
        \item Clock offsets obtained by integrating clock drifts.
        \item \Cref{fig:clock-drift-derivation,fig:residual-clock-drift-derivation}
    \end{compactitem}

    \item
    Sensor model:
    \begin{compactitem}
        \item Range error:
        \begin{compactitem}
            \item \(\mathrm{rangeError} = \mathrm{correctedPseudorange} - \mathrm{geometricRange}\)
            \item Correcting pseudorange for
                receiver clock offset + residual clock offset,
                satellite clock offset,
                tropospheric errors.
            \item Ionospheric errors not corrected explicitly (see \Cref{sec:impl-corrections}).
        \end{compactitem}
        \item Velocity error:
        \begin{compactitem}
            \item \(\mathrm{velocityError} = \mathrm{correctedVelocity} - \mathrm{velocityDifference}\)
            \item Correcting velocity for receiver clock drift + residual clock drift.
            \item Encountered a bug in \sirf (see \Cref{sec:impl-carrier-freq}).
        \end{compactitem}
        \item Thresholding
        \begin{compactitem}
            \item Not using the PDF model for very large errors, instead using
                  fixed probability of exceeding threshold.
            \item Threshold \SI{\pseudorangeThreshold}{\meter}, probability \num{\pseudorangeThresholdProbability} for pseudorange.
            \item Threshold \SI{\velocityThreshold}{\meter\per\second}, probability \num{\velocityThresholdProbability} for velocity measurements.
        \end{compactitem}

        \item Probabilities for measurements within the thresholds:
        \begin{compactitem}
            \item Using PDF of normal distribution.
            \item Range errors:
                  \(\normalDistribution(\mu = \SI{\pseudorangeErrorMu}{\meter}, \sigma = \SI{\pseudorangeErrorSigma}{\meter})\)
            \item Velocity errors:
                  \(\normalDistribution(\mu = \SI{\velocityErrorMu}{\meter\per\second}, \sigma = \SI{\velocityErrorSigma}{\meter\per\second})\)
            \item Better fitting probability distribution, or switching probability distributions are possible improvements.
            \item \Cref{fig:pseudorange-errors,fig:pseudorange-hist,fig:velocity-hist}
        \end{compactitem}
    \end{compactitem}

    \item
    Similar precision to the simple model so far, possible improvements.

    \item
    \Cref{algo:gps-measurement-domain,sec:measurement-domain}.
\end{compactitem}
\end{compactitem}
